% tikz/reversi-board.tex
% Reversi/Othello board (8x8) as reusable TikZ macros (no tikzpicture).
% Convention: A1 is top-left. Internal indices: A1 -> (0,7).

% ==========================================================
% Parameters (cell units / cell fractions)
% - Anything used in "circle(...)" is in cell units (like \rvMoveRad).
% - Line widths and node sizes need TeX lengths -> derived from cell length.
% ==========================================================
\providecommand{\rvBoardColor}{green!35}

% Geometry (CELL UNITS)
% Stones are drawn like moves: radius in cell units.
\providecommand{\rvStoneRad}{0.4}      % 0.50 => diameter exactly 1 cell
\providecommand{\rvMoveRad}{0.4}       % move marker radius (cell units)
\providecommand{\rvLastMoveRad}{0.1}   % last-move dot radius (cell units)

% Line widths (CELL FRACTIONS of one cell)
\providecommand{\rvOuterLWFrac}{0.2}
\providecommand{\rvGridLWFrac}{0.1}
\providecommand{\rvStoneLWFrac}{0.4}
\providecommand{\rvMoveLWFrac}{0.4}
\providecommand{\rvLastMoveLWFrac}{0.4}
\providecommand{\rvMarkLWFrac}{1.0}

% Flip marker (CELL UNITS + line width as fraction)
\providecommand{\rvFlipLineLWFrac}{1.0}
\providecommand{\rvFlipYmin}{0.38}
\providecommand{\rvFlipYmax}{0.62}
\providecommand{\rvFlipTriH}{0.13}
\providecommand{\rvFlipTriW}{0.12}

% Text
\providecommand{\rvCoordFont}{\small}
\providecommand{\rvValueScale}{0.95}

% ==========================================================
% Internal lengths (defined once even if file is input multiple times)
% ==========================================================
\makeatletter
\@ifundefined{rvCellLen}{\newlength{\rvCellLen}}{}
\@ifundefined{rvOuterLW}{\newlength{\rvOuterLW}}{}
\@ifundefined{rvGridLW}{\newlength{\rvGridLW}}{}
\@ifundefined{rvStoneLW}{\newlength{\rvStoneLW}}{}
\@ifundefined{rvMoveLW}{\newlength{\rvMoveLW}}{}
\@ifundefined{rvLastMoveLW}{\newlength{\rvLastMoveLW}}{}
\@ifundefined{rvMarkLW}{\newlength{\rvMarkLW}}{}
\@ifundefined{rvFlipLineLW}{\newlength{\rvFlipLineLW}}{}
\makeatother

% ==========================================================
% Scale derivation (recomputed when drawing a board)
% ==========================================================
\providecommand{\rvSetupScale}{%
	\pgfextractx{\rvCellLen}{\pgfpoint{1}{0}}%
	\pgfmathsetlength{\rvOuterLW}{\rvOuterLWFrac*\rvCellLen}%
	\pgfmathsetlength{\rvGridLW}{\rvGridLWFrac*\rvCellLen}%
	\pgfmathsetlength{\rvStoneLW}{\rvStoneLWFrac*\rvCellLen}%
	\pgfmathsetlength{\rvMoveLW}{\rvMoveLWFrac*\rvCellLen}%
	\pgfmathsetlength{\rvLastMoveLW}{\rvLastMoveLWFrac*\rvCellLen}%
	\pgfmathsetlength{\rvMarkLW}{\rvMarkLWFrac*\rvCellLen}%
	\pgfmathsetlength{\rvFlipLineLW}{\rvFlipLineLWFrac*\rvCellLen}%
}

% ==========================================================
% Board
% ==========================================================
\providecommand{\rvBoard}{%
	\rvSetupScale%
	\def\N{8}%
	\fill[\rvBoardColor] (0,0) rectangle (\N,\N);
	\draw[line width=\rvOuterLW] (0,0) rectangle (\N,\N);
	\foreach \i in {1,...,7}{%
		\draw[line width=\rvGridLW] (\i,0) -- (\i,\N);
		\draw[line width=\rvGridLW] (0,\i) -- (\N,\i);
	}%
}

\providecommand{\rvCoords}{%
	\def\N{8}%
	\foreach \x/\lab in {1/A,2/B,3/C,4/D,5/E,6/F,7/G,8/H}{%
		\node[font=\rvCoordFont] at (\x-0.5,\N+0.65) {\lab};
	}%
	\foreach \r in {1,...,8}{%
		\node[font=\rvCoordFont] at (-0.65,\N-\r+0.5) {\r};
	}%
}

% ==========================================================
% Internal primitives (indices 0..7)
% Stones are drawn in cell units (like \rvMoveRad).
% ==========================================================
\providecommand{\rvStoneWhite}[2]{%
	\rvSetupScale%
	\path[draw=black, fill=white, line width=\rvStoneLW]
	(#1+0.5,#2+0.5) circle (\rvStoneRad);
}
\providecommand{\rvStoneBlack}[2]{%
	\rvSetupScale%
	\path[draw=black, fill=black, line width=\rvStoneLW]
	(#1+0.5,#2+0.5) circle (\rvStoneRad);
}

\providecommand{\rvMoveWhite}[2]{%
	\rvSetupScale%
	\draw[dashed, draw=black, line width=\rvMoveLW]
	(#1+0.5,#2+0.5) circle (\rvMoveRad);
}
\providecommand{\rvMoveBlack}[2]{%
	\rvSetupScale%
	\draw[draw=black, line width=\rvMoveLW]
	(#1+0.5,#2+0.5) circle (\rvMoveRad);
}

\providecommand{\rvMarkFrame}[2]{%
	\rvSetupScale%
	\draw[draw=red, line width=\rvMarkLW]
	(#1,#2) rectangle ++(1,1);
}

\providecommand{\rvLastMoveDot}[2]{%
	\rvSetupScale%
	\draw[draw=yellow!70!orange, fill=yellow!80!orange, line width=\rvLastMoveLW]
	(#1+0.5,#2+0.5) circle (\rvLastMoveRad);
}

\providecommand{\rvFlipSymbol}[2]{%
	\rvSetupScale%
	\draw[draw=yellow!70!orange, line width=\rvFlipLineLW, line cap=round]
	(#1+0.5, #2+\rvFlipYmin) -- (#1+0.5, #2+\rvFlipYmax);
	\path[draw=yellow!70!orange, fill=yellow!70!orange]
	(#1+0.5, #2+\rvFlipYmax+\rvFlipTriH) --
	(#1+0.5-\rvFlipTriW, #2+\rvFlipYmax) --
	(#1+0.5+\rvFlipTriW, #2+\rvFlipYmax) -- cycle;
	\path[draw=yellow!70!orange, fill=yellow!70!orange]
	(#1+0.5, #2+\rvFlipYmin-\rvFlipTriH) --
	(#1+0.5-\rvFlipTriW, #2+\rvFlipYmin) --
	(#1+0.5+\rvFlipTriW, #2+\rvFlipYmin) -- cycle;
}

\providecommand{\rvValueLabel}[3]{%
	\node[scale=\rvValueScale] at (#1+0.5,#2+0.5) {#3};
}

% ==========================================================
% Mapping: file/rank -> internal indices
% ==========================================================
\providecommand{\rvFileToX}[1]{%
	\ifnum\pdfstrcmp{#1}{A}=0 0\else
	\ifnum\pdfstrcmp{#1}{B}=0 1\else
	\ifnum\pdfstrcmp{#1}{C}=0 2\else
	\ifnum\pdfstrcmp{#1}{D}=0 3\else
	\ifnum\pdfstrcmp{#1}{E}=0 4\else
	\ifnum\pdfstrcmp{#1}{F}=0 5\else
	\ifnum\pdfstrcmp{#1}{G}=0 6\else
	\ifnum\pdfstrcmp{#1}{H}=0 7\else
	\ifnum\pdfstrcmp{#1}{a}=0 0\else
	\ifnum\pdfstrcmp{#1}{b}=0 1\else
	\ifnum\pdfstrcmp{#1}{c}=0 2\else
	\ifnum\pdfstrcmp{#1}{d}=0 3\else
	\ifnum\pdfstrcmp{#1}{e}=0 4\else
	\ifnum\pdfstrcmp{#1}{f}=0 5\else
	\ifnum\pdfstrcmp{#1}{g}=0 6\else
	\ifnum\pdfstrcmp{#1}{h}=0 7\else
	-1%
	\fi\fi\fi\fi\fi\fi\fi\fi
	\fi\fi\fi\fi\fi\fi\fi\fi
}
\providecommand{\rvRankToY}[1]{\numexpr8-#1\relax}

% ==========================================================
% User-facing API
% ==========================================================
\providecommand{\rvStoneBlackAt}[2]{\rvStoneBlack{\rvFileToX{#1}}{\rvRankToY{#2}}}
\providecommand{\rvStoneWhiteAt}[2]{\rvStoneWhite{\rvFileToX{#1}}{\rvRankToY{#2}}}
\providecommand{\rvMoveBlackAt}[2]{\rvMoveBlack{\rvFileToX{#1}}{\rvRankToY{#2}}}
\providecommand{\rvMoveWhiteAt}[2]{\rvMoveWhite{\rvFileToX{#1}}{\rvRankToY{#2}}}
\providecommand{\rvMarkFrameAt}[2]{\rvMarkFrame{\rvFileToX{#1}}{\rvRankToY{#2}}}

\providecommand{\rvStonesBlack}[1]{\foreach \p in {#1}{\expandafter\rvStoneBlackAux\p\relax}}
\providecommand{\rvStonesWhite}[1]{\foreach \p in {#1}{\expandafter\rvStoneWhiteAux\p\relax}}
\providecommand{\rvMovesBlack}[1]{\foreach \p in {#1}{\expandafter\rvMoveBlackAux\p\relax}}
\providecommand{\rvMovesWhite}[1]{\foreach \p in {#1}{\expandafter\rvMoveWhiteAux\p\relax}}
\providecommand{\rvMarkFrames}[1]{\foreach \p in {#1}{\expandafter\rvMarkFrameAux\p\relax}}
\providecommand{\rvFlips}[1]{\foreach \p in {#1}{\expandafter\rvFlipAux\p\relax}}

\providecommand{\rvLastMove}[1]{\expandafter\rvLastMoveAux#1\relax}
\def\rvLastMoveAux#1#2\relax{%
	\rvLastMoveDot{\rvFileToX{#1}}{\rvRankToY{#2}}%
}

\providecommand{\rvValueMap}[1]{\foreach \pv in {#1}{\expandafter\rvValueMapAux\pv\relax}}
\def\rvValueMapAux#1:#2\relax{%
	\expandafter\rvValueMapCoordAux#1\relax{#2}%
}
\def\rvValueMapCoordAux#1#2\relax#3{%
	\rvValueLabel{\rvFileToX{#1}}{\rvRankToY{#2}}{#3}%
}

% Coordinate parser for tokens like "E4"
\def\rvStoneBlackAux#1#2\relax{\rvStoneBlackAt{#1}{#2}}
\def\rvStoneWhiteAux#1#2\relax{\rvStoneWhiteAt{#1}{#2}}
\def\rvMoveBlackAux#1#2\relax{\rvMoveBlackAt{#1}{#2}}
\def\rvMoveWhiteAux#1#2\relax{\rvMoveWhiteAt{#1}{#2}}
\def\rvMarkFrameAux#1#2\relax{\rvMarkFrameAt{#1}{#2}}
\def\rvFlipAux#1#2\relax{\rvFlipSymbol{\rvFileToX{#1}}{\rvRankToY{#2}}}

% Matrix input to display values on board
\providecommand{\rvValueMatrix}[8]{%
	\rvValueMatrixRow{#1}{7}%
	\rvValueMatrixRow{#2}{6}%
	\rvValueMatrixRow{#3}{5}%
	\rvValueMatrixRow{#4}{4}%
	\rvValueMatrixRow{#5}{3}%
	\rvValueMatrixRow{#6}{2}%
	\rvValueMatrixRow{#7}{1}%
	\rvValueMatrixRow{#8}{0}%
}

\def\rvValueMatrixRow#1#2{%
	\foreach \v [count=\x from 0] in {#1}{%
		\rvValueLabel{\x}{#2}{\v}%
	}%
}
