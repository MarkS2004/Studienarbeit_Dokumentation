\chapter{Theoretische Grundlagen}
\label{cha:Grundlagen}

\todo[inline]{Lorem ipsum dolor sit amet, consetetur sadipscing elitr, sed diam nonumy eirmod tempor invidunt ut labore et dolore magna aliquyam erat, sed diam voluptua. At vero eos et accusam et justo duo dolores et ea rebum. Stet clita kasd gubergren, no sea takimata sanctus est Lorem ipsum dolor sit amet. Lorem ipsum dolor sit amet, consetetur sadipscing elitr, sed diam nonumy eirmod tempor invidunt ut labore et dolore magna aliquyam erat, sed diam voluptua. At vero eos et accusam et justo duo dolores et ea rebum. Stet clita kasd gubergren, no sea takimata sanctus est Lorem ipsum dolor sit amet.}

\section{Einführung in Spielaufbau und Regeln von Othello}
\label{sec:othello}
Bei Othello handelt es sich um ein strategisches Brettspiel für zwei Personen, bei dem die Spielenden abwechselnd schwarze und weiße Spielsteine auf einem schachbrettartigen Spielfeld platzieren. Das Spiel wurde in den 1880er-Jahren entwickelt und unter dem Namen \textit{Reversi} (lat.: \glqq die Umgedrehten\grqq) bekannt. Othello stellt die moderne und heute bedeutendere Variante von Reversi dar, deren Regelwerk standardisiert ist und bei internationalen Turnieren angewendet wird. Othello wurde 1971 in Japan patentiert und unterscheidet sich von Reversi insbesondere durch eine fest definierte Startaufstellung der vier Anfangsspielsteine sowie durch keine Begrenzung der Spielsteinnutzung. \autocite{beppimenozziBriefHistoryOthello2009}

Für die Entwicklung eines Othello spielenden Roboters sowie des zugehörigen, hardwarelimitierten Spielalgorithmus ist ein grundlegendes Verständnis von Spielaufbau, Spielablauf und Regelwerk erforderlich. Im Folgenden werden daher die für diese Arbeit geltenden Spielregeln dargestellt. Die Beschreibung orientiert sich am offiziellen Regelwerk \autocite{worldothellofederationOfficialRulesGame} der World Othello Federation, wobei Detailregeln je nach Herausgeber bzw. Veranstalter variieren können. Ergänzend wird die in dieser Arbeit verwendete Darstellung des Spielbretts erläutert.

\paragraph{Ziel des Spiels}
Ziel des Spiels ist es, am Spielende mehr Spielsteine der eigenen Farbe auf dem Spielbrett zu besitzen als der Gegenspieler. Das Spiel endet, sobald keiner der beiden Spieler gemäß den Regeln einen gültigen Zug ausführen kann.

\paragraph{Spielaufbau und Startaufstellung}
Das Spielbrett besteht aus $8\times 8 = 64$ Feldern. Zu Beginn erhält jeder Spieler 32 Spielsteine und wählt eine der beiden Farben Schwarz oder Weiß. Verfügt ein Spieler im weiteren Spielverlauf vorübergehend über keine eigenen Spielsteine mehr, ist jedoch weiterhin ein regelkonformer Zug möglich, so werden ihm die erforderlichen Spielsteine vom gegnerischen Spieler bereitgestellt; die Anzahl der nutzbaren Spielsteine ist somit faktisch unbegrenzt. Im Folgenden wird der Spieler mit den weißen Spielsteinen als \emph{Weiß} und der Spieler mit den schwarzen Spielsteinen als \emph{Schwarz} bezeichnet. Zur eindeutigen Beschreibung der Platzierung von Spielsteinen wird ein Koordinatensystem eingeführt, das entlang einer Achse die Zahlen 1 bis 8 und entlang der anderen die Buchstaben A bis H verwendet. Abbildung \ref{fig:othello-startaufstellung} zeigt dieses Koordinatensystem einschließlich der initialen Startaufstellung der Spielsteine.

\begin{figure}[hbt]
	\centering
	
	% tikz/reversi-board.tex
% Reversi/Othello board (8x8) as reusable TikZ macros (no tikzpicture).
% Convention: A1 is top-left. Internal indices: A1 -> (0,7).

% ==========================================================
% Parameters (cell units / cell fractions)
% - Anything used in "circle(...)" is in cell units (like \rvMoveRad).
% - Line widths and node sizes need TeX lengths -> derived from cell length.
% ==========================================================
\providecommand{\rvBoardColor}{green!35}

% Geometry (CELL UNITS)
% Stones are drawn like moves: radius in cell units.
\providecommand{\rvStoneRad}{0.4}      % 0.50 => diameter exactly 1 cell
\providecommand{\rvMoveRad}{0.4}       % move marker radius (cell units)
\providecommand{\rvLastMoveRad}{0.1}   % last-move dot radius (cell units)

% Line widths (CELL FRACTIONS of one cell)
\providecommand{\rvOuterLWFrac}{0.2}
\providecommand{\rvGridLWFrac}{0.1}
\providecommand{\rvStoneLWFrac}{0.4}
\providecommand{\rvMoveLWFrac}{0.4}
\providecommand{\rvLastMoveLWFrac}{0.4}
\providecommand{\rvMarkLWFrac}{1.0}

% Flip marker (CELL UNITS + line width as fraction)
\providecommand{\rvFlipLineLWFrac}{1.0}
\providecommand{\rvFlipYmin}{0.38}
\providecommand{\rvFlipYmax}{0.62}
\providecommand{\rvFlipTriH}{0.13}
\providecommand{\rvFlipTriW}{0.12}

% Text
\providecommand{\rvCoordFont}{\small}
\providecommand{\rvValueScale}{0.95}

% ==========================================================
% Internal lengths (defined once even if file is input multiple times)
% ==========================================================
\makeatletter
\@ifundefined{rvCellLen}{\newlength{\rvCellLen}}{}
\@ifundefined{rvOuterLW}{\newlength{\rvOuterLW}}{}
\@ifundefined{rvGridLW}{\newlength{\rvGridLW}}{}
\@ifundefined{rvStoneLW}{\newlength{\rvStoneLW}}{}
\@ifundefined{rvMoveLW}{\newlength{\rvMoveLW}}{}
\@ifundefined{rvLastMoveLW}{\newlength{\rvLastMoveLW}}{}
\@ifundefined{rvMarkLW}{\newlength{\rvMarkLW}}{}
\@ifundefined{rvFlipLineLW}{\newlength{\rvFlipLineLW}}{}
\makeatother

% ==========================================================
% Scale derivation (recomputed when drawing a board)
% ==========================================================
\providecommand{\rvSetupScale}{%
	\pgfextractx{\rvCellLen}{\pgfpoint{1}{0}}%
	\pgfmathsetlength{\rvOuterLW}{\rvOuterLWFrac*\rvCellLen}%
	\pgfmathsetlength{\rvGridLW}{\rvGridLWFrac*\rvCellLen}%
	\pgfmathsetlength{\rvStoneLW}{\rvStoneLWFrac*\rvCellLen}%
	\pgfmathsetlength{\rvMoveLW}{\rvMoveLWFrac*\rvCellLen}%
	\pgfmathsetlength{\rvLastMoveLW}{\rvLastMoveLWFrac*\rvCellLen}%
	\pgfmathsetlength{\rvMarkLW}{\rvMarkLWFrac*\rvCellLen}%
	\pgfmathsetlength{\rvFlipLineLW}{\rvFlipLineLWFrac*\rvCellLen}%
}

% ==========================================================
% Board
% ==========================================================
\providecommand{\rvBoard}{%
	\rvSetupScale%
	\def\N{8}%
	\fill[\rvBoardColor] (0,0) rectangle (\N,\N);
	\draw[line width=\rvOuterLW] (0,0) rectangle (\N,\N);
	\foreach \i in {1,...,7}{%
		\draw[line width=\rvGridLW] (\i,0) -- (\i,\N);
		\draw[line width=\rvGridLW] (0,\i) -- (\N,\i);
	}%
}

\providecommand{\rvCoords}{%
	\def\N{8}%
	\foreach \x/\lab in {1/A,2/B,3/C,4/D,5/E,6/F,7/G,8/H}{%
		\node[font=\rvCoordFont] at (\x-0.5,\N+0.65) {\lab};
	}%
	\foreach \r in {1,...,8}{%
		\node[font=\rvCoordFont] at (-0.65,\N-\r+0.5) {\r};
	}%
}

% ==========================================================
% Internal primitives (indices 0..7)
% Stones are drawn in cell units (like \rvMoveRad).
% ==========================================================
\providecommand{\rvStoneWhite}[2]{%
	\rvSetupScale%
	\path[draw=black, fill=white, line width=\rvStoneLW]
	(#1+0.5,#2+0.5) circle (\rvStoneRad);
}
\providecommand{\rvStoneBlack}[2]{%
	\rvSetupScale%
	\path[draw=black, fill=black, line width=\rvStoneLW]
	(#1+0.5,#2+0.5) circle (\rvStoneRad);
}

\providecommand{\rvMoveWhite}[2]{%
	\rvSetupScale%
	\draw[dashed, draw=black, line width=\rvMoveLW]
	(#1+0.5,#2+0.5) circle (\rvMoveRad);
}
\providecommand{\rvMoveBlack}[2]{%
	\rvSetupScale%
	\draw[draw=black, line width=\rvMoveLW]
	(#1+0.5,#2+0.5) circle (\rvMoveRad);
}

\providecommand{\rvMarkFrame}[2]{%
	\rvSetupScale%
	\draw[draw=red, line width=\rvMarkLW]
	(#1,#2) rectangle ++(1,1);
}

\providecommand{\rvLastMoveDot}[2]{%
	\rvSetupScale%
	\draw[draw=yellow!70!orange, fill=yellow!80!orange, line width=\rvLastMoveLW]
	(#1+0.5,#2+0.5) circle (\rvLastMoveRad);
}

\providecommand{\rvFlipSymbol}[2]{%
	\rvSetupScale%
	\draw[draw=yellow!70!orange, line width=\rvFlipLineLW, line cap=round]
	(#1+0.5, #2+\rvFlipYmin) -- (#1+0.5, #2+\rvFlipYmax);
	\path[draw=yellow!70!orange, fill=yellow!70!orange]
	(#1+0.5, #2+\rvFlipYmax+\rvFlipTriH) --
	(#1+0.5-\rvFlipTriW, #2+\rvFlipYmax) --
	(#1+0.5+\rvFlipTriW, #2+\rvFlipYmax) -- cycle;
	\path[draw=yellow!70!orange, fill=yellow!70!orange]
	(#1+0.5, #2+\rvFlipYmin-\rvFlipTriH) --
	(#1+0.5-\rvFlipTriW, #2+\rvFlipYmin) --
	(#1+0.5+\rvFlipTriW, #2+\rvFlipYmin) -- cycle;
}

\providecommand{\rvValueLabel}[3]{%
	\node[scale=\rvValueScale] at (#1+0.5,#2+0.5) {#3};
}

% ==========================================================
% Mapping: file/rank -> internal indices
% ==========================================================
\providecommand{\rvFileToX}[1]{%
	\ifnum\pdfstrcmp{#1}{A}=0 0\else
	\ifnum\pdfstrcmp{#1}{B}=0 1\else
	\ifnum\pdfstrcmp{#1}{C}=0 2\else
	\ifnum\pdfstrcmp{#1}{D}=0 3\else
	\ifnum\pdfstrcmp{#1}{E}=0 4\else
	\ifnum\pdfstrcmp{#1}{F}=0 5\else
	\ifnum\pdfstrcmp{#1}{G}=0 6\else
	\ifnum\pdfstrcmp{#1}{H}=0 7\else
	\ifnum\pdfstrcmp{#1}{a}=0 0\else
	\ifnum\pdfstrcmp{#1}{b}=0 1\else
	\ifnum\pdfstrcmp{#1}{c}=0 2\else
	\ifnum\pdfstrcmp{#1}{d}=0 3\else
	\ifnum\pdfstrcmp{#1}{e}=0 4\else
	\ifnum\pdfstrcmp{#1}{f}=0 5\else
	\ifnum\pdfstrcmp{#1}{g}=0 6\else
	\ifnum\pdfstrcmp{#1}{h}=0 7\else
	-1%
	\fi\fi\fi\fi\fi\fi\fi\fi
	\fi\fi\fi\fi\fi\fi\fi\fi
}
\providecommand{\rvRankToY}[1]{\numexpr8-#1\relax}

% ==========================================================
% User-facing API
% ==========================================================
\providecommand{\rvStoneBlackAt}[2]{\rvStoneBlack{\rvFileToX{#1}}{\rvRankToY{#2}}}
\providecommand{\rvStoneWhiteAt}[2]{\rvStoneWhite{\rvFileToX{#1}}{\rvRankToY{#2}}}
\providecommand{\rvMoveBlackAt}[2]{\rvMoveBlack{\rvFileToX{#1}}{\rvRankToY{#2}}}
\providecommand{\rvMoveWhiteAt}[2]{\rvMoveWhite{\rvFileToX{#1}}{\rvRankToY{#2}}}
\providecommand{\rvMarkFrameAt}[2]{\rvMarkFrame{\rvFileToX{#1}}{\rvRankToY{#2}}}

\providecommand{\rvStonesBlack}[1]{\foreach \p in {#1}{\expandafter\rvStoneBlackAux\p\relax}}
\providecommand{\rvStonesWhite}[1]{\foreach \p in {#1}{\expandafter\rvStoneWhiteAux\p\relax}}
\providecommand{\rvMovesBlack}[1]{\foreach \p in {#1}{\expandafter\rvMoveBlackAux\p\relax}}
\providecommand{\rvMovesWhite}[1]{\foreach \p in {#1}{\expandafter\rvMoveWhiteAux\p\relax}}
\providecommand{\rvMarkFrames}[1]{\foreach \p in {#1}{\expandafter\rvMarkFrameAux\p\relax}}
\providecommand{\rvFlips}[1]{\foreach \p in {#1}{\expandafter\rvFlipAux\p\relax}}

\providecommand{\rvLastMove}[1]{\expandafter\rvLastMoveAux#1\relax}
\def\rvLastMoveAux#1#2\relax{%
	\rvLastMoveDot{\rvFileToX{#1}}{\rvRankToY{#2}}%
}

\providecommand{\rvValueMap}[1]{\foreach \pv in {#1}{\expandafter\rvValueMapAux\pv\relax}}
\def\rvValueMapAux#1:#2\relax{%
	\expandafter\rvValueMapCoordAux#1\relax{#2}%
}
\def\rvValueMapCoordAux#1#2\relax#3{%
	\rvValueLabel{\rvFileToX{#1}}{\rvRankToY{#2}}{#3}%
}

% Coordinate parser for tokens like "E4"
\def\rvStoneBlackAux#1#2\relax{\rvStoneBlackAt{#1}{#2}}
\def\rvStoneWhiteAux#1#2\relax{\rvStoneWhiteAt{#1}{#2}}
\def\rvMoveBlackAux#1#2\relax{\rvMoveBlackAt{#1}{#2}}
\def\rvMoveWhiteAux#1#2\relax{\rvMoveWhiteAt{#1}{#2}}
\def\rvMarkFrameAux#1#2\relax{\rvMarkFrameAt{#1}{#2}}
\def\rvFlipAux#1#2\relax{\rvFlipSymbol{\rvFileToX{#1}}{\rvRankToY{#2}}}

% Matrix input to display values on board
\providecommand{\rvValueMatrix}[8]{%
	\rvValueMatrixRow{#1}{7}%
	\rvValueMatrixRow{#2}{6}%
	\rvValueMatrixRow{#3}{5}%
	\rvValueMatrixRow{#4}{4}%
	\rvValueMatrixRow{#5}{3}%
	\rvValueMatrixRow{#6}{2}%
	\rvValueMatrixRow{#7}{1}%
	\rvValueMatrixRow{#8}{0}%
}

\def\rvValueMatrixRow#1#2{%
	\foreach \v [count=\x from 0] in {#1}{%
		\rvValueLabel{\x}{#2}{\v}%
	}%
}

	
	\subfigure[Startaufstellung bei Othello]{
		\begin{tikzpicture}[scale=0.6]
			\rvBoard
			\rvCoords
			\rvStonesBlack{E4, D5}
			\rvStonesWhite{E5, D4}
		\end{tikzpicture}
		\label{fig:othello-startaufstellung-regelkonform}
	}
	\hspace{12mm}
	\subfigure[Nicht zulässige Startaufstellung]{
		\begin{tikzpicture}[scale=0.6]
			\rvBoard
			\rvCoords
			\rvStonesBlack{D4, D5}
			\rvStonesWhite{E4, E5}
		\end{tikzpicture}
		\label{fig:othello-startaufstellung-unzulaessig}
	}
	
	\caption[Startaufstellung bei Othello]{Regelkonforme (a) und nicht zulässige (b) Startaufstellungen des Spiels Othello; letztere ist jedoch gemäß den Regeln von Reversi zulässig.}
	\label{fig:othello-startaufstellung}
\end{figure}

Zu Beginn werden vier Spielsteine, zwei weiße und zwei schwarze, gemäß Abbildung \ref{fig:othello-startaufstellung-regelkonform} in der Mitte des Spielbretts diagonal angeordnet. Dabei sind die weißen Spielsteine aus Sicht des jeweiligen Spielers auf der rechten Seite zu platzieren. Eine andere initiale Anordnung, wie in Abbildung \ref{fig:othello-startaufstellung-unzulaessig} dargestellt, ist im Spiel Othello im Gegensatz zu Reversi nicht zulässig.

\paragraph{Spielzug}
Ein Spielzug besteht darin, einen oder mehrere gegnerische Spielsteine zu \textit{überflügeln}, sodass diese in die eigene Farbe umgedreht werden. Überflügeln bedeutet, einen Spielstein so zu platzieren, dass eine oder mehrere zusammenhängende Reihen gegnerischer Spielsteine zwischen zwei eigenen Spielsteinen eingeschlossen werden. Eine solche Reihe kann aus einem oder mehreren Spielsteinen bestehen und vertikal, horizontal oder diagonal auf dem Spielbrett verlaufen, sofern sie eine durchgehende Linie bildet. Zur Verdeutlichung dient das in Abbildung \ref{fig:othello-ueberfluegeln} dargestellte, konstruierte Beispiel.

\begin{figure}[hbt]
	\centering
	
	% tikz/reversi-board.tex
% Reversi/Othello board (8x8) as reusable TikZ macros (no tikzpicture).
% Convention: A1 is top-left. Internal indices: A1 -> (0,7).

% ==========================================================
% Parameters (cell units / cell fractions)
% - Anything used in "circle(...)" is in cell units (like \rvMoveRad).
% - Line widths and node sizes need TeX lengths -> derived from cell length.
% ==========================================================
\providecommand{\rvBoardColor}{green!35}

% Geometry (CELL UNITS)
% Stones are drawn like moves: radius in cell units.
\providecommand{\rvStoneRad}{0.4}      % 0.50 => diameter exactly 1 cell
\providecommand{\rvMoveRad}{0.4}       % move marker radius (cell units)
\providecommand{\rvLastMoveRad}{0.1}   % last-move dot radius (cell units)

% Line widths (CELL FRACTIONS of one cell)
\providecommand{\rvOuterLWFrac}{0.2}
\providecommand{\rvGridLWFrac}{0.1}
\providecommand{\rvStoneLWFrac}{0.4}
\providecommand{\rvMoveLWFrac}{0.4}
\providecommand{\rvLastMoveLWFrac}{0.4}
\providecommand{\rvMarkLWFrac}{1.0}

% Flip marker (CELL UNITS + line width as fraction)
\providecommand{\rvFlipLineLWFrac}{1.0}
\providecommand{\rvFlipYmin}{0.38}
\providecommand{\rvFlipYmax}{0.62}
\providecommand{\rvFlipTriH}{0.13}
\providecommand{\rvFlipTriW}{0.12}

% Text
\providecommand{\rvCoordFont}{\small}
\providecommand{\rvValueScale}{0.95}

% ==========================================================
% Internal lengths (defined once even if file is input multiple times)
% ==========================================================
\makeatletter
\@ifundefined{rvCellLen}{\newlength{\rvCellLen}}{}
\@ifundefined{rvOuterLW}{\newlength{\rvOuterLW}}{}
\@ifundefined{rvGridLW}{\newlength{\rvGridLW}}{}
\@ifundefined{rvStoneLW}{\newlength{\rvStoneLW}}{}
\@ifundefined{rvMoveLW}{\newlength{\rvMoveLW}}{}
\@ifundefined{rvLastMoveLW}{\newlength{\rvLastMoveLW}}{}
\@ifundefined{rvMarkLW}{\newlength{\rvMarkLW}}{}
\@ifundefined{rvFlipLineLW}{\newlength{\rvFlipLineLW}}{}
\makeatother

% ==========================================================
% Scale derivation (recomputed when drawing a board)
% ==========================================================
\providecommand{\rvSetupScale}{%
	\pgfextractx{\rvCellLen}{\pgfpoint{1}{0}}%
	\pgfmathsetlength{\rvOuterLW}{\rvOuterLWFrac*\rvCellLen}%
	\pgfmathsetlength{\rvGridLW}{\rvGridLWFrac*\rvCellLen}%
	\pgfmathsetlength{\rvStoneLW}{\rvStoneLWFrac*\rvCellLen}%
	\pgfmathsetlength{\rvMoveLW}{\rvMoveLWFrac*\rvCellLen}%
	\pgfmathsetlength{\rvLastMoveLW}{\rvLastMoveLWFrac*\rvCellLen}%
	\pgfmathsetlength{\rvMarkLW}{\rvMarkLWFrac*\rvCellLen}%
	\pgfmathsetlength{\rvFlipLineLW}{\rvFlipLineLWFrac*\rvCellLen}%
}

% ==========================================================
% Board
% ==========================================================
\providecommand{\rvBoard}{%
	\rvSetupScale%
	\def\N{8}%
	\fill[\rvBoardColor] (0,0) rectangle (\N,\N);
	\draw[line width=\rvOuterLW] (0,0) rectangle (\N,\N);
	\foreach \i in {1,...,7}{%
		\draw[line width=\rvGridLW] (\i,0) -- (\i,\N);
		\draw[line width=\rvGridLW] (0,\i) -- (\N,\i);
	}%
}

\providecommand{\rvCoords}{%
	\def\N{8}%
	\foreach \x/\lab in {1/A,2/B,3/C,4/D,5/E,6/F,7/G,8/H}{%
		\node[font=\rvCoordFont] at (\x-0.5,\N+0.65) {\lab};
	}%
	\foreach \r in {1,...,8}{%
		\node[font=\rvCoordFont] at (-0.65,\N-\r+0.5) {\r};
	}%
}

% ==========================================================
% Internal primitives (indices 0..7)
% Stones are drawn in cell units (like \rvMoveRad).
% ==========================================================
\providecommand{\rvStoneWhite}[2]{%
	\rvSetupScale%
	\path[draw=black, fill=white, line width=\rvStoneLW]
	(#1+0.5,#2+0.5) circle (\rvStoneRad);
}
\providecommand{\rvStoneBlack}[2]{%
	\rvSetupScale%
	\path[draw=black, fill=black, line width=\rvStoneLW]
	(#1+0.5,#2+0.5) circle (\rvStoneRad);
}

\providecommand{\rvMoveWhite}[2]{%
	\rvSetupScale%
	\draw[dashed, draw=black, line width=\rvMoveLW]
	(#1+0.5,#2+0.5) circle (\rvMoveRad);
}
\providecommand{\rvMoveBlack}[2]{%
	\rvSetupScale%
	\draw[draw=black, line width=\rvMoveLW]
	(#1+0.5,#2+0.5) circle (\rvMoveRad);
}

\providecommand{\rvMarkFrame}[2]{%
	\rvSetupScale%
	\draw[draw=red, line width=\rvMarkLW]
	(#1,#2) rectangle ++(1,1);
}

\providecommand{\rvLastMoveDot}[2]{%
	\rvSetupScale%
	\draw[draw=yellow!70!orange, fill=yellow!80!orange, line width=\rvLastMoveLW]
	(#1+0.5,#2+0.5) circle (\rvLastMoveRad);
}

\providecommand{\rvFlipSymbol}[2]{%
	\rvSetupScale%
	\draw[draw=yellow!70!orange, line width=\rvFlipLineLW, line cap=round]
	(#1+0.5, #2+\rvFlipYmin) -- (#1+0.5, #2+\rvFlipYmax);
	\path[draw=yellow!70!orange, fill=yellow!70!orange]
	(#1+0.5, #2+\rvFlipYmax+\rvFlipTriH) --
	(#1+0.5-\rvFlipTriW, #2+\rvFlipYmax) --
	(#1+0.5+\rvFlipTriW, #2+\rvFlipYmax) -- cycle;
	\path[draw=yellow!70!orange, fill=yellow!70!orange]
	(#1+0.5, #2+\rvFlipYmin-\rvFlipTriH) --
	(#1+0.5-\rvFlipTriW, #2+\rvFlipYmin) --
	(#1+0.5+\rvFlipTriW, #2+\rvFlipYmin) -- cycle;
}

\providecommand{\rvValueLabel}[3]{%
	\node[scale=\rvValueScale] at (#1+0.5,#2+0.5) {#3};
}

% ==========================================================
% Mapping: file/rank -> internal indices
% ==========================================================
\providecommand{\rvFileToX}[1]{%
	\ifnum\pdfstrcmp{#1}{A}=0 0\else
	\ifnum\pdfstrcmp{#1}{B}=0 1\else
	\ifnum\pdfstrcmp{#1}{C}=0 2\else
	\ifnum\pdfstrcmp{#1}{D}=0 3\else
	\ifnum\pdfstrcmp{#1}{E}=0 4\else
	\ifnum\pdfstrcmp{#1}{F}=0 5\else
	\ifnum\pdfstrcmp{#1}{G}=0 6\else
	\ifnum\pdfstrcmp{#1}{H}=0 7\else
	\ifnum\pdfstrcmp{#1}{a}=0 0\else
	\ifnum\pdfstrcmp{#1}{b}=0 1\else
	\ifnum\pdfstrcmp{#1}{c}=0 2\else
	\ifnum\pdfstrcmp{#1}{d}=0 3\else
	\ifnum\pdfstrcmp{#1}{e}=0 4\else
	\ifnum\pdfstrcmp{#1}{f}=0 5\else
	\ifnum\pdfstrcmp{#1}{g}=0 6\else
	\ifnum\pdfstrcmp{#1}{h}=0 7\else
	-1%
	\fi\fi\fi\fi\fi\fi\fi\fi
	\fi\fi\fi\fi\fi\fi\fi\fi
}
\providecommand{\rvRankToY}[1]{\numexpr8-#1\relax}

% ==========================================================
% User-facing API
% ==========================================================
\providecommand{\rvStoneBlackAt}[2]{\rvStoneBlack{\rvFileToX{#1}}{\rvRankToY{#2}}}
\providecommand{\rvStoneWhiteAt}[2]{\rvStoneWhite{\rvFileToX{#1}}{\rvRankToY{#2}}}
\providecommand{\rvMoveBlackAt}[2]{\rvMoveBlack{\rvFileToX{#1}}{\rvRankToY{#2}}}
\providecommand{\rvMoveWhiteAt}[2]{\rvMoveWhite{\rvFileToX{#1}}{\rvRankToY{#2}}}
\providecommand{\rvMarkFrameAt}[2]{\rvMarkFrame{\rvFileToX{#1}}{\rvRankToY{#2}}}

\providecommand{\rvStonesBlack}[1]{\foreach \p in {#1}{\expandafter\rvStoneBlackAux\p\relax}}
\providecommand{\rvStonesWhite}[1]{\foreach \p in {#1}{\expandafter\rvStoneWhiteAux\p\relax}}
\providecommand{\rvMovesBlack}[1]{\foreach \p in {#1}{\expandafter\rvMoveBlackAux\p\relax}}
\providecommand{\rvMovesWhite}[1]{\foreach \p in {#1}{\expandafter\rvMoveWhiteAux\p\relax}}
\providecommand{\rvMarkFrames}[1]{\foreach \p in {#1}{\expandafter\rvMarkFrameAux\p\relax}}
\providecommand{\rvFlips}[1]{\foreach \p in {#1}{\expandafter\rvFlipAux\p\relax}}

\providecommand{\rvLastMove}[1]{\expandafter\rvLastMoveAux#1\relax}
\def\rvLastMoveAux#1#2\relax{%
	\rvLastMoveDot{\rvFileToX{#1}}{\rvRankToY{#2}}%
}

\providecommand{\rvValueMap}[1]{\foreach \pv in {#1}{\expandafter\rvValueMapAux\pv\relax}}
\def\rvValueMapAux#1:#2\relax{%
	\expandafter\rvValueMapCoordAux#1\relax{#2}%
}
\def\rvValueMapCoordAux#1#2\relax#3{%
	\rvValueLabel{\rvFileToX{#1}}{\rvRankToY{#2}}{#3}%
}

% Coordinate parser for tokens like "E4"
\def\rvStoneBlackAux#1#2\relax{\rvStoneBlackAt{#1}{#2}}
\def\rvStoneWhiteAux#1#2\relax{\rvStoneWhiteAt{#1}{#2}}
\def\rvMoveBlackAux#1#2\relax{\rvMoveBlackAt{#1}{#2}}
\def\rvMoveWhiteAux#1#2\relax{\rvMoveWhiteAt{#1}{#2}}
\def\rvMarkFrameAux#1#2\relax{\rvMarkFrameAt{#1}{#2}}
\def\rvFlipAux#1#2\relax{\rvFlipSymbol{\rvFileToX{#1}}{\rvRankToY{#2}}}

% Matrix input to display values on board
\providecommand{\rvValueMatrix}[8]{%
	\rvValueMatrixRow{#1}{7}%
	\rvValueMatrixRow{#2}{6}%
	\rvValueMatrixRow{#3}{5}%
	\rvValueMatrixRow{#4}{4}%
	\rvValueMatrixRow{#5}{3}%
	\rvValueMatrixRow{#6}{2}%
	\rvValueMatrixRow{#7}{1}%
	\rvValueMatrixRow{#8}{0}%
}

\def\rvValueMatrixRow#1#2{%
	\foreach \v [count=\x from 0] in {#1}{%
		\rvValueLabel{\x}{#2}{\v}%
	}%
}

	
	\subfigure[Spielsituation vor dem Zug]{
		\begin{tikzpicture}[scale=0.6]
			\rvBoard
			\rvCoords
			\rvStonesBlack{C4, D4, E4, F4}
			\rvStonesWhite{B4}
			\rvMovesWhite{G4}
		\end{tikzpicture}
		\label{fig:othello-ueberfluegeln-vor-dem-zug}
	}
	\hspace{12mm}
	\subfigure[Spielsituation nach dem Zug]{
		\begin{tikzpicture}[scale=0.6]
			\rvBoard
			\rvCoords
			\rvStonesWhite{B4, C4, D4, E4, F4, G4}
			\rvLastMove{G4}
			\rvFlips{C4, D4, E4, F4}
		\end{tikzpicture}
		\label{fig:othello-ueberfluegeln-nach-dem-zug}
	}
	
	\caption[Darstellung eines Spielzugs \glqq Überflügeln\grqq]{Darstellung eines Spielzugs \glqq Überflügeln\grqq{} vor (a) und nach dem Zug (b).}
	\label{fig:othello-ueberfluegeln}
\end{figure}

In Abbildung \ref{fig:othello-ueberfluegeln-vor-dem-zug} ist die Spielsituation vor dem Überflügeln dargestellt. Weiß ist am Zug und besitzt genau eine regelkonforme Zugmöglichkeit: das Platzieren eines Spielsteins auf Feld G4. Diese Zugmöglichkeit ist in den folgenden Abbildungen durch einen gestrichelten Kreis gekennzeichnet; bei Schwarz am Zug wird der Kreis durchgezogen dargestellt.\\
Abbildung \ref{fig:othello-ueberfluegeln-nach-dem-zug} zeigt die Spielsituation nach dem Platzieren des weißen Spielsteins auf dem Feld G4. Der zuletzt ausgeführte Spielzug ist durch eine kreisförmige Markierung auf dem entsprechenden Spielstein hervorgehoben. Infolge dieses Zuges werden die dazwischenliegenden gegnerischen Spielsteine auf den Feldern C4, D4, E4 und E5 umgedreht. Dieser Vorgang ist durch Pfeile auf den betroffenen Spielsteinen dargestellt.

\paragraph{Spielregeln}
Es gelten die folgenden Regeln:
\vspace{-0.9\baselineskip}
\begin{enumerate}[
	leftmargin=*,      % bündig am linken Satzspiegel
	labelsep=0.6em,    % Abstand Nummer → Text
	itemsep=0.5em,     % Abstand zwischen Items (sehr kompakt)
	topsep=0pt,      % Abstand vor/nach der Liste
	parsep=0pt,
	partopsep=0pt
	]
	
	\item Schwarz zieht immer zuerst.
	
	\item Kann ein Spieler in seinem Zug keinen gegnerischen Spielstein überflügeln und umdrehen, verfällt sein Zug und der Gegenspieler ist erneut am Zug. Steht mindestens eine regelkonforme Zugmöglichkeit zur Verfügung, ist der Zug auszuführen.
	
	\item Überflügelte gegnerische Spielsteine können in allen Richtungen umgedreht werden, in denen sie überflügelt werden. Dies umfasst waagerechte, senkrechte und diagonale Richtungen. Ein Spielzug kann dabei mehrere Reihen gleichzeitig betreffen.
	Ein Beispiel ist in Abbildung \ref{fig:othello-ueberfluegeln-mehrerer-Reihen} dargestellt, wobei durch das Platzieren eines weißen Spielsteins auf dem Feld C7 mehrere gegnerische Reihen gleichzeitig umgedreht werden.
	\begin{figure}[!h]
		\centering
		% tikz/reversi-board.tex
% Reversi/Othello board (8x8) as reusable TikZ macros (no tikzpicture).
% Convention: A1 is top-left. Internal indices: A1 -> (0,7).

% ==========================================================
% Parameters (cell units / cell fractions)
% - Anything used in "circle(...)" is in cell units (like \rvMoveRad).
% - Line widths and node sizes need TeX lengths -> derived from cell length.
% ==========================================================
\providecommand{\rvBoardColor}{green!35}

% Geometry (CELL UNITS)
% Stones are drawn like moves: radius in cell units.
\providecommand{\rvStoneRad}{0.4}      % 0.50 => diameter exactly 1 cell
\providecommand{\rvMoveRad}{0.4}       % move marker radius (cell units)
\providecommand{\rvLastMoveRad}{0.1}   % last-move dot radius (cell units)

% Line widths (CELL FRACTIONS of one cell)
\providecommand{\rvOuterLWFrac}{0.2}
\providecommand{\rvGridLWFrac}{0.1}
\providecommand{\rvStoneLWFrac}{0.4}
\providecommand{\rvMoveLWFrac}{0.4}
\providecommand{\rvLastMoveLWFrac}{0.4}
\providecommand{\rvMarkLWFrac}{1.0}

% Flip marker (CELL UNITS + line width as fraction)
\providecommand{\rvFlipLineLWFrac}{1.0}
\providecommand{\rvFlipYmin}{0.38}
\providecommand{\rvFlipYmax}{0.62}
\providecommand{\rvFlipTriH}{0.13}
\providecommand{\rvFlipTriW}{0.12}

% Text
\providecommand{\rvCoordFont}{\small}
\providecommand{\rvValueScale}{0.95}

% ==========================================================
% Internal lengths (defined once even if file is input multiple times)
% ==========================================================
\makeatletter
\@ifundefined{rvCellLen}{\newlength{\rvCellLen}}{}
\@ifundefined{rvOuterLW}{\newlength{\rvOuterLW}}{}
\@ifundefined{rvGridLW}{\newlength{\rvGridLW}}{}
\@ifundefined{rvStoneLW}{\newlength{\rvStoneLW}}{}
\@ifundefined{rvMoveLW}{\newlength{\rvMoveLW}}{}
\@ifundefined{rvLastMoveLW}{\newlength{\rvLastMoveLW}}{}
\@ifundefined{rvMarkLW}{\newlength{\rvMarkLW}}{}
\@ifundefined{rvFlipLineLW}{\newlength{\rvFlipLineLW}}{}
\makeatother

% ==========================================================
% Scale derivation (recomputed when drawing a board)
% ==========================================================
\providecommand{\rvSetupScale}{%
	\pgfextractx{\rvCellLen}{\pgfpoint{1}{0}}%
	\pgfmathsetlength{\rvOuterLW}{\rvOuterLWFrac*\rvCellLen}%
	\pgfmathsetlength{\rvGridLW}{\rvGridLWFrac*\rvCellLen}%
	\pgfmathsetlength{\rvStoneLW}{\rvStoneLWFrac*\rvCellLen}%
	\pgfmathsetlength{\rvMoveLW}{\rvMoveLWFrac*\rvCellLen}%
	\pgfmathsetlength{\rvLastMoveLW}{\rvLastMoveLWFrac*\rvCellLen}%
	\pgfmathsetlength{\rvMarkLW}{\rvMarkLWFrac*\rvCellLen}%
	\pgfmathsetlength{\rvFlipLineLW}{\rvFlipLineLWFrac*\rvCellLen}%
}

% ==========================================================
% Board
% ==========================================================
\providecommand{\rvBoard}{%
	\rvSetupScale%
	\def\N{8}%
	\fill[\rvBoardColor] (0,0) rectangle (\N,\N);
	\draw[line width=\rvOuterLW] (0,0) rectangle (\N,\N);
	\foreach \i in {1,...,7}{%
		\draw[line width=\rvGridLW] (\i,0) -- (\i,\N);
		\draw[line width=\rvGridLW] (0,\i) -- (\N,\i);
	}%
}

\providecommand{\rvCoords}{%
	\def\N{8}%
	\foreach \x/\lab in {1/A,2/B,3/C,4/D,5/E,6/F,7/G,8/H}{%
		\node[font=\rvCoordFont] at (\x-0.5,\N+0.65) {\lab};
	}%
	\foreach \r in {1,...,8}{%
		\node[font=\rvCoordFont] at (-0.65,\N-\r+0.5) {\r};
	}%
}

% ==========================================================
% Internal primitives (indices 0..7)
% Stones are drawn in cell units (like \rvMoveRad).
% ==========================================================
\providecommand{\rvStoneWhite}[2]{%
	\rvSetupScale%
	\path[draw=black, fill=white, line width=\rvStoneLW]
	(#1+0.5,#2+0.5) circle (\rvStoneRad);
}
\providecommand{\rvStoneBlack}[2]{%
	\rvSetupScale%
	\path[draw=black, fill=black, line width=\rvStoneLW]
	(#1+0.5,#2+0.5) circle (\rvStoneRad);
}

\providecommand{\rvMoveWhite}[2]{%
	\rvSetupScale%
	\draw[dashed, draw=black, line width=\rvMoveLW]
	(#1+0.5,#2+0.5) circle (\rvMoveRad);
}
\providecommand{\rvMoveBlack}[2]{%
	\rvSetupScale%
	\draw[draw=black, line width=\rvMoveLW]
	(#1+0.5,#2+0.5) circle (\rvMoveRad);
}

\providecommand{\rvMarkFrame}[2]{%
	\rvSetupScale%
	\draw[draw=red, line width=\rvMarkLW]
	(#1,#2) rectangle ++(1,1);
}

\providecommand{\rvLastMoveDot}[2]{%
	\rvSetupScale%
	\draw[draw=yellow!70!orange, fill=yellow!80!orange, line width=\rvLastMoveLW]
	(#1+0.5,#2+0.5) circle (\rvLastMoveRad);
}

\providecommand{\rvFlipSymbol}[2]{%
	\rvSetupScale%
	\draw[draw=yellow!70!orange, line width=\rvFlipLineLW, line cap=round]
	(#1+0.5, #2+\rvFlipYmin) -- (#1+0.5, #2+\rvFlipYmax);
	\path[draw=yellow!70!orange, fill=yellow!70!orange]
	(#1+0.5, #2+\rvFlipYmax+\rvFlipTriH) --
	(#1+0.5-\rvFlipTriW, #2+\rvFlipYmax) --
	(#1+0.5+\rvFlipTriW, #2+\rvFlipYmax) -- cycle;
	\path[draw=yellow!70!orange, fill=yellow!70!orange]
	(#1+0.5, #2+\rvFlipYmin-\rvFlipTriH) --
	(#1+0.5-\rvFlipTriW, #2+\rvFlipYmin) --
	(#1+0.5+\rvFlipTriW, #2+\rvFlipYmin) -- cycle;
}

\providecommand{\rvValueLabel}[3]{%
	\node[scale=\rvValueScale] at (#1+0.5,#2+0.5) {#3};
}

% ==========================================================
% Mapping: file/rank -> internal indices
% ==========================================================
\providecommand{\rvFileToX}[1]{%
	\ifnum\pdfstrcmp{#1}{A}=0 0\else
	\ifnum\pdfstrcmp{#1}{B}=0 1\else
	\ifnum\pdfstrcmp{#1}{C}=0 2\else
	\ifnum\pdfstrcmp{#1}{D}=0 3\else
	\ifnum\pdfstrcmp{#1}{E}=0 4\else
	\ifnum\pdfstrcmp{#1}{F}=0 5\else
	\ifnum\pdfstrcmp{#1}{G}=0 6\else
	\ifnum\pdfstrcmp{#1}{H}=0 7\else
	\ifnum\pdfstrcmp{#1}{a}=0 0\else
	\ifnum\pdfstrcmp{#1}{b}=0 1\else
	\ifnum\pdfstrcmp{#1}{c}=0 2\else
	\ifnum\pdfstrcmp{#1}{d}=0 3\else
	\ifnum\pdfstrcmp{#1}{e}=0 4\else
	\ifnum\pdfstrcmp{#1}{f}=0 5\else
	\ifnum\pdfstrcmp{#1}{g}=0 6\else
	\ifnum\pdfstrcmp{#1}{h}=0 7\else
	-1%
	\fi\fi\fi\fi\fi\fi\fi\fi
	\fi\fi\fi\fi\fi\fi\fi\fi
}
\providecommand{\rvRankToY}[1]{\numexpr8-#1\relax}

% ==========================================================
% User-facing API
% ==========================================================
\providecommand{\rvStoneBlackAt}[2]{\rvStoneBlack{\rvFileToX{#1}}{\rvRankToY{#2}}}
\providecommand{\rvStoneWhiteAt}[2]{\rvStoneWhite{\rvFileToX{#1}}{\rvRankToY{#2}}}
\providecommand{\rvMoveBlackAt}[2]{\rvMoveBlack{\rvFileToX{#1}}{\rvRankToY{#2}}}
\providecommand{\rvMoveWhiteAt}[2]{\rvMoveWhite{\rvFileToX{#1}}{\rvRankToY{#2}}}
\providecommand{\rvMarkFrameAt}[2]{\rvMarkFrame{\rvFileToX{#1}}{\rvRankToY{#2}}}

\providecommand{\rvStonesBlack}[1]{\foreach \p in {#1}{\expandafter\rvStoneBlackAux\p\relax}}
\providecommand{\rvStonesWhite}[1]{\foreach \p in {#1}{\expandafter\rvStoneWhiteAux\p\relax}}
\providecommand{\rvMovesBlack}[1]{\foreach \p in {#1}{\expandafter\rvMoveBlackAux\p\relax}}
\providecommand{\rvMovesWhite}[1]{\foreach \p in {#1}{\expandafter\rvMoveWhiteAux\p\relax}}
\providecommand{\rvMarkFrames}[1]{\foreach \p in {#1}{\expandafter\rvMarkFrameAux\p\relax}}
\providecommand{\rvFlips}[1]{\foreach \p in {#1}{\expandafter\rvFlipAux\p\relax}}

\providecommand{\rvLastMove}[1]{\expandafter\rvLastMoveAux#1\relax}
\def\rvLastMoveAux#1#2\relax{%
	\rvLastMoveDot{\rvFileToX{#1}}{\rvRankToY{#2}}%
}

\providecommand{\rvValueMap}[1]{\foreach \pv in {#1}{\expandafter\rvValueMapAux\pv\relax}}
\def\rvValueMapAux#1:#2\relax{%
	\expandafter\rvValueMapCoordAux#1\relax{#2}%
}
\def\rvValueMapCoordAux#1#2\relax#3{%
	\rvValueLabel{\rvFileToX{#1}}{\rvRankToY{#2}}{#3}%
}

% Coordinate parser for tokens like "E4"
\def\rvStoneBlackAux#1#2\relax{\rvStoneBlackAt{#1}{#2}}
\def\rvStoneWhiteAux#1#2\relax{\rvStoneWhiteAt{#1}{#2}}
\def\rvMoveBlackAux#1#2\relax{\rvMoveBlackAt{#1}{#2}}
\def\rvMoveWhiteAux#1#2\relax{\rvMoveWhiteAt{#1}{#2}}
\def\rvMarkFrameAux#1#2\relax{\rvMarkFrameAt{#1}{#2}}
\def\rvFlipAux#1#2\relax{\rvFlipSymbol{\rvFileToX{#1}}{\rvRankToY{#2}}}

% Matrix input to display values on board
\providecommand{\rvValueMatrix}[8]{%
	\rvValueMatrixRow{#1}{7}%
	\rvValueMatrixRow{#2}{6}%
	\rvValueMatrixRow{#3}{5}%
	\rvValueMatrixRow{#4}{4}%
	\rvValueMatrixRow{#5}{3}%
	\rvValueMatrixRow{#6}{2}%
	\rvValueMatrixRow{#7}{1}%
	\rvValueMatrixRow{#8}{0}%
}

\def\rvValueMatrixRow#1#2{%
	\foreach \v [count=\x from 0] in {#1}{%
		\rvValueLabel{\x}{#2}{\v}%
	}%
}

		\subfigure[Spielsituation vor dem Zug]{
			\begin{tikzpicture}[scale=0.6]
				\rvBoard
				\rvCoords
				\rvStonesBlack{C4, C5, C6, D6, D7, E5, F4}
				\rvStonesWhite{C3, E7, G3}
				\rvMovesWhite{C7}
			\end{tikzpicture}
		}
		\hspace{12mm}
		\subfigure[Spielsituation nach dem Zug]{
			\begin{tikzpicture}[scale=0.6]
				\rvBoard
				\rvCoords
				\rvStonesWhite{C7, C3, E7, G3, C4, C5, C6, D6, D7, E5, F4}
				\rvLastMove{C7}
				\rvFlips{C4, C5, C6, D6, D7, E5, F4}
			\end{tikzpicture}
		}
		\caption[Spielzug mit Überflügelung mehrerer Spielsteinreihen]{Spielzug mit Überflügelung mehrerer Spielsteinreihen vor (a) und nach dem Zug (b).}
		\label{fig:othello-ueberfluegeln-mehrerer-Reihen}
	\end{figure}
	
	\item Eigene Spielsteine dürfen nicht übersprungen werden, um gegnerische Spielsteine zu überflügeln. Eine Überflügelung ist nur zulässig, wenn zwischen dem neu platzierten und einem eigenen Spielstein ausschließlich gegnerische Spielsteine liegen.
	Ein Beispiel ist in Abbildung \ref{fig:othello-ueberspringen-eigene-reihe} dargestellt, wobei durch das Platzieren eines schwarzen Spielsteins auf dem Feld A7 lediglich der direkt angrenzende gegnerische Spielstein umgedreht wird.
	\begin{figure}[!h]
		\centering
		% tikz/reversi-board.tex
% Reversi/Othello board (8x8) as reusable TikZ macros (no tikzpicture).
% Convention: A1 is top-left. Internal indices: A1 -> (0,7).

% ==========================================================
% Parameters (cell units / cell fractions)
% - Anything used in "circle(...)" is in cell units (like \rvMoveRad).
% - Line widths and node sizes need TeX lengths -> derived from cell length.
% ==========================================================
\providecommand{\rvBoardColor}{green!35}

% Geometry (CELL UNITS)
% Stones are drawn like moves: radius in cell units.
\providecommand{\rvStoneRad}{0.4}      % 0.50 => diameter exactly 1 cell
\providecommand{\rvMoveRad}{0.4}       % move marker radius (cell units)
\providecommand{\rvLastMoveRad}{0.1}   % last-move dot radius (cell units)

% Line widths (CELL FRACTIONS of one cell)
\providecommand{\rvOuterLWFrac}{0.2}
\providecommand{\rvGridLWFrac}{0.1}
\providecommand{\rvStoneLWFrac}{0.4}
\providecommand{\rvMoveLWFrac}{0.4}
\providecommand{\rvLastMoveLWFrac}{0.4}
\providecommand{\rvMarkLWFrac}{1.0}

% Flip marker (CELL UNITS + line width as fraction)
\providecommand{\rvFlipLineLWFrac}{1.0}
\providecommand{\rvFlipYmin}{0.38}
\providecommand{\rvFlipYmax}{0.62}
\providecommand{\rvFlipTriH}{0.13}
\providecommand{\rvFlipTriW}{0.12}

% Text
\providecommand{\rvCoordFont}{\small}
\providecommand{\rvValueScale}{0.95}

% ==========================================================
% Internal lengths (defined once even if file is input multiple times)
% ==========================================================
\makeatletter
\@ifundefined{rvCellLen}{\newlength{\rvCellLen}}{}
\@ifundefined{rvOuterLW}{\newlength{\rvOuterLW}}{}
\@ifundefined{rvGridLW}{\newlength{\rvGridLW}}{}
\@ifundefined{rvStoneLW}{\newlength{\rvStoneLW}}{}
\@ifundefined{rvMoveLW}{\newlength{\rvMoveLW}}{}
\@ifundefined{rvLastMoveLW}{\newlength{\rvLastMoveLW}}{}
\@ifundefined{rvMarkLW}{\newlength{\rvMarkLW}}{}
\@ifundefined{rvFlipLineLW}{\newlength{\rvFlipLineLW}}{}
\makeatother

% ==========================================================
% Scale derivation (recomputed when drawing a board)
% ==========================================================
\providecommand{\rvSetupScale}{%
	\pgfextractx{\rvCellLen}{\pgfpoint{1}{0}}%
	\pgfmathsetlength{\rvOuterLW}{\rvOuterLWFrac*\rvCellLen}%
	\pgfmathsetlength{\rvGridLW}{\rvGridLWFrac*\rvCellLen}%
	\pgfmathsetlength{\rvStoneLW}{\rvStoneLWFrac*\rvCellLen}%
	\pgfmathsetlength{\rvMoveLW}{\rvMoveLWFrac*\rvCellLen}%
	\pgfmathsetlength{\rvLastMoveLW}{\rvLastMoveLWFrac*\rvCellLen}%
	\pgfmathsetlength{\rvMarkLW}{\rvMarkLWFrac*\rvCellLen}%
	\pgfmathsetlength{\rvFlipLineLW}{\rvFlipLineLWFrac*\rvCellLen}%
}

% ==========================================================
% Board
% ==========================================================
\providecommand{\rvBoard}{%
	\rvSetupScale%
	\def\N{8}%
	\fill[\rvBoardColor] (0,0) rectangle (\N,\N);
	\draw[line width=\rvOuterLW] (0,0) rectangle (\N,\N);
	\foreach \i in {1,...,7}{%
		\draw[line width=\rvGridLW] (\i,0) -- (\i,\N);
		\draw[line width=\rvGridLW] (0,\i) -- (\N,\i);
	}%
}

\providecommand{\rvCoords}{%
	\def\N{8}%
	\foreach \x/\lab in {1/A,2/B,3/C,4/D,5/E,6/F,7/G,8/H}{%
		\node[font=\rvCoordFont] at (\x-0.5,\N+0.65) {\lab};
	}%
	\foreach \r in {1,...,8}{%
		\node[font=\rvCoordFont] at (-0.65,\N-\r+0.5) {\r};
	}%
}

% ==========================================================
% Internal primitives (indices 0..7)
% Stones are drawn in cell units (like \rvMoveRad).
% ==========================================================
\providecommand{\rvStoneWhite}[2]{%
	\rvSetupScale%
	\path[draw=black, fill=white, line width=\rvStoneLW]
	(#1+0.5,#2+0.5) circle (\rvStoneRad);
}
\providecommand{\rvStoneBlack}[2]{%
	\rvSetupScale%
	\path[draw=black, fill=black, line width=\rvStoneLW]
	(#1+0.5,#2+0.5) circle (\rvStoneRad);
}

\providecommand{\rvMoveWhite}[2]{%
	\rvSetupScale%
	\draw[dashed, draw=black, line width=\rvMoveLW]
	(#1+0.5,#2+0.5) circle (\rvMoveRad);
}
\providecommand{\rvMoveBlack}[2]{%
	\rvSetupScale%
	\draw[draw=black, line width=\rvMoveLW]
	(#1+0.5,#2+0.5) circle (\rvMoveRad);
}

\providecommand{\rvMarkFrame}[2]{%
	\rvSetupScale%
	\draw[draw=red, line width=\rvMarkLW]
	(#1,#2) rectangle ++(1,1);
}

\providecommand{\rvLastMoveDot}[2]{%
	\rvSetupScale%
	\draw[draw=yellow!70!orange, fill=yellow!80!orange, line width=\rvLastMoveLW]
	(#1+0.5,#2+0.5) circle (\rvLastMoveRad);
}

\providecommand{\rvFlipSymbol}[2]{%
	\rvSetupScale%
	\draw[draw=yellow!70!orange, line width=\rvFlipLineLW, line cap=round]
	(#1+0.5, #2+\rvFlipYmin) -- (#1+0.5, #2+\rvFlipYmax);
	\path[draw=yellow!70!orange, fill=yellow!70!orange]
	(#1+0.5, #2+\rvFlipYmax+\rvFlipTriH) --
	(#1+0.5-\rvFlipTriW, #2+\rvFlipYmax) --
	(#1+0.5+\rvFlipTriW, #2+\rvFlipYmax) -- cycle;
	\path[draw=yellow!70!orange, fill=yellow!70!orange]
	(#1+0.5, #2+\rvFlipYmin-\rvFlipTriH) --
	(#1+0.5-\rvFlipTriW, #2+\rvFlipYmin) --
	(#1+0.5+\rvFlipTriW, #2+\rvFlipYmin) -- cycle;
}

\providecommand{\rvValueLabel}[3]{%
	\node[scale=\rvValueScale] at (#1+0.5,#2+0.5) {#3};
}

% ==========================================================
% Mapping: file/rank -> internal indices
% ==========================================================
\providecommand{\rvFileToX}[1]{%
	\ifnum\pdfstrcmp{#1}{A}=0 0\else
	\ifnum\pdfstrcmp{#1}{B}=0 1\else
	\ifnum\pdfstrcmp{#1}{C}=0 2\else
	\ifnum\pdfstrcmp{#1}{D}=0 3\else
	\ifnum\pdfstrcmp{#1}{E}=0 4\else
	\ifnum\pdfstrcmp{#1}{F}=0 5\else
	\ifnum\pdfstrcmp{#1}{G}=0 6\else
	\ifnum\pdfstrcmp{#1}{H}=0 7\else
	\ifnum\pdfstrcmp{#1}{a}=0 0\else
	\ifnum\pdfstrcmp{#1}{b}=0 1\else
	\ifnum\pdfstrcmp{#1}{c}=0 2\else
	\ifnum\pdfstrcmp{#1}{d}=0 3\else
	\ifnum\pdfstrcmp{#1}{e}=0 4\else
	\ifnum\pdfstrcmp{#1}{f}=0 5\else
	\ifnum\pdfstrcmp{#1}{g}=0 6\else
	\ifnum\pdfstrcmp{#1}{h}=0 7\else
	-1%
	\fi\fi\fi\fi\fi\fi\fi\fi
	\fi\fi\fi\fi\fi\fi\fi\fi
}
\providecommand{\rvRankToY}[1]{\numexpr8-#1\relax}

% ==========================================================
% User-facing API
% ==========================================================
\providecommand{\rvStoneBlackAt}[2]{\rvStoneBlack{\rvFileToX{#1}}{\rvRankToY{#2}}}
\providecommand{\rvStoneWhiteAt}[2]{\rvStoneWhite{\rvFileToX{#1}}{\rvRankToY{#2}}}
\providecommand{\rvMoveBlackAt}[2]{\rvMoveBlack{\rvFileToX{#1}}{\rvRankToY{#2}}}
\providecommand{\rvMoveWhiteAt}[2]{\rvMoveWhite{\rvFileToX{#1}}{\rvRankToY{#2}}}
\providecommand{\rvMarkFrameAt}[2]{\rvMarkFrame{\rvFileToX{#1}}{\rvRankToY{#2}}}

\providecommand{\rvStonesBlack}[1]{\foreach \p in {#1}{\expandafter\rvStoneBlackAux\p\relax}}
\providecommand{\rvStonesWhite}[1]{\foreach \p in {#1}{\expandafter\rvStoneWhiteAux\p\relax}}
\providecommand{\rvMovesBlack}[1]{\foreach \p in {#1}{\expandafter\rvMoveBlackAux\p\relax}}
\providecommand{\rvMovesWhite}[1]{\foreach \p in {#1}{\expandafter\rvMoveWhiteAux\p\relax}}
\providecommand{\rvMarkFrames}[1]{\foreach \p in {#1}{\expandafter\rvMarkFrameAux\p\relax}}
\providecommand{\rvFlips}[1]{\foreach \p in {#1}{\expandafter\rvFlipAux\p\relax}}

\providecommand{\rvLastMove}[1]{\expandafter\rvLastMoveAux#1\relax}
\def\rvLastMoveAux#1#2\relax{%
	\rvLastMoveDot{\rvFileToX{#1}}{\rvRankToY{#2}}%
}

\providecommand{\rvValueMap}[1]{\foreach \pv in {#1}{\expandafter\rvValueMapAux\pv\relax}}
\def\rvValueMapAux#1:#2\relax{%
	\expandafter\rvValueMapCoordAux#1\relax{#2}%
}
\def\rvValueMapCoordAux#1#2\relax#3{%
	\rvValueLabel{\rvFileToX{#1}}{\rvRankToY{#2}}{#3}%
}

% Coordinate parser for tokens like "E4"
\def\rvStoneBlackAux#1#2\relax{\rvStoneBlackAt{#1}{#2}}
\def\rvStoneWhiteAux#1#2\relax{\rvStoneWhiteAt{#1}{#2}}
\def\rvMoveBlackAux#1#2\relax{\rvMoveBlackAt{#1}{#2}}
\def\rvMoveWhiteAux#1#2\relax{\rvMoveWhiteAt{#1}{#2}}
\def\rvMarkFrameAux#1#2\relax{\rvMarkFrameAt{#1}{#2}}
\def\rvFlipAux#1#2\relax{\rvFlipSymbol{\rvFileToX{#1}}{\rvRankToY{#2}}}

% Matrix input to display values on board
\providecommand{\rvValueMatrix}[8]{%
	\rvValueMatrixRow{#1}{7}%
	\rvValueMatrixRow{#2}{6}%
	\rvValueMatrixRow{#3}{5}%
	\rvValueMatrixRow{#4}{4}%
	\rvValueMatrixRow{#5}{3}%
	\rvValueMatrixRow{#6}{2}%
	\rvValueMatrixRow{#7}{1}%
	\rvValueMatrixRow{#8}{0}%
}

\def\rvValueMatrixRow#1#2{%
	\foreach \v [count=\x from 0] in {#1}{%
		\rvValueLabel{\x}{#2}{\v}%
	}%
}

		\subfigure[Spielsituation vor dem Zug]{
			\begin{tikzpicture}[scale=0.6]
				\rvBoard
				\rvCoords
				\rvStonesBlack{C7, F7}
				\rvStonesWhite{B7, D7, E7}
				\rvMovesBlack{A7}
			\end{tikzpicture}
		}
		\hspace{12mm}
		\subfigure[Spielsituation nach dem Zug]{
			\begin{tikzpicture}[scale=0.6]
				\rvBoard
				\rvCoords
				\rvStonesBlack{A7, C7, F7, B7}
				\rvStonesWhite{D7, E7}
				\rvFlips{B7}
				\rvLastMove{A7}
			\end{tikzpicture}
		}
		\caption[Spielzug ohne Überflügelung über eigene Spielsteine]{Spielzug ohne Überflügelung über eigene Spielsteine vor (a) und nach dem Zug (b).}
		\label{fig:othello-ueberspringen-eigene-reihe}
	\end{figure}
	
	\item Ein Spielstein darf nur infolge eines Spielzugs überflügelt werden und muss dabei in direkter Linie zum neu platzierten Spielstein liegen. Spielsteine außerhalb dieser Linie bleiben unbeeinflusst.
	Ein Beispiel ist in Abbildung \ref{fig:othello-ueberfluegeln-in-direkter-linie} dargestellt, wobei durch das Platzieren eines schwarzen Spielsteins auf dem Feld A2 ausschließlich die Spielsteine A3 und A4 umgedreht werden.
	
	\begin{figure}[!h]
		\centering
		% tikz/reversi-board.tex
% Reversi/Othello board (8x8) as reusable TikZ macros (no tikzpicture).
% Convention: A1 is top-left. Internal indices: A1 -> (0,7).

% ==========================================================
% Parameters (cell units / cell fractions)
% - Anything used in "circle(...)" is in cell units (like \rvMoveRad).
% - Line widths and node sizes need TeX lengths -> derived from cell length.
% ==========================================================
\providecommand{\rvBoardColor}{green!35}

% Geometry (CELL UNITS)
% Stones are drawn like moves: radius in cell units.
\providecommand{\rvStoneRad}{0.4}      % 0.50 => diameter exactly 1 cell
\providecommand{\rvMoveRad}{0.4}       % move marker radius (cell units)
\providecommand{\rvLastMoveRad}{0.1}   % last-move dot radius (cell units)

% Line widths (CELL FRACTIONS of one cell)
\providecommand{\rvOuterLWFrac}{0.2}
\providecommand{\rvGridLWFrac}{0.1}
\providecommand{\rvStoneLWFrac}{0.4}
\providecommand{\rvMoveLWFrac}{0.4}
\providecommand{\rvLastMoveLWFrac}{0.4}
\providecommand{\rvMarkLWFrac}{1.0}

% Flip marker (CELL UNITS + line width as fraction)
\providecommand{\rvFlipLineLWFrac}{1.0}
\providecommand{\rvFlipYmin}{0.38}
\providecommand{\rvFlipYmax}{0.62}
\providecommand{\rvFlipTriH}{0.13}
\providecommand{\rvFlipTriW}{0.12}

% Text
\providecommand{\rvCoordFont}{\small}
\providecommand{\rvValueScale}{0.95}

% ==========================================================
% Internal lengths (defined once even if file is input multiple times)
% ==========================================================
\makeatletter
\@ifundefined{rvCellLen}{\newlength{\rvCellLen}}{}
\@ifundefined{rvOuterLW}{\newlength{\rvOuterLW}}{}
\@ifundefined{rvGridLW}{\newlength{\rvGridLW}}{}
\@ifundefined{rvStoneLW}{\newlength{\rvStoneLW}}{}
\@ifundefined{rvMoveLW}{\newlength{\rvMoveLW}}{}
\@ifundefined{rvLastMoveLW}{\newlength{\rvLastMoveLW}}{}
\@ifundefined{rvMarkLW}{\newlength{\rvMarkLW}}{}
\@ifundefined{rvFlipLineLW}{\newlength{\rvFlipLineLW}}{}
\makeatother

% ==========================================================
% Scale derivation (recomputed when drawing a board)
% ==========================================================
\providecommand{\rvSetupScale}{%
	\pgfextractx{\rvCellLen}{\pgfpoint{1}{0}}%
	\pgfmathsetlength{\rvOuterLW}{\rvOuterLWFrac*\rvCellLen}%
	\pgfmathsetlength{\rvGridLW}{\rvGridLWFrac*\rvCellLen}%
	\pgfmathsetlength{\rvStoneLW}{\rvStoneLWFrac*\rvCellLen}%
	\pgfmathsetlength{\rvMoveLW}{\rvMoveLWFrac*\rvCellLen}%
	\pgfmathsetlength{\rvLastMoveLW}{\rvLastMoveLWFrac*\rvCellLen}%
	\pgfmathsetlength{\rvMarkLW}{\rvMarkLWFrac*\rvCellLen}%
	\pgfmathsetlength{\rvFlipLineLW}{\rvFlipLineLWFrac*\rvCellLen}%
}

% ==========================================================
% Board
% ==========================================================
\providecommand{\rvBoard}{%
	\rvSetupScale%
	\def\N{8}%
	\fill[\rvBoardColor] (0,0) rectangle (\N,\N);
	\draw[line width=\rvOuterLW] (0,0) rectangle (\N,\N);
	\foreach \i in {1,...,7}{%
		\draw[line width=\rvGridLW] (\i,0) -- (\i,\N);
		\draw[line width=\rvGridLW] (0,\i) -- (\N,\i);
	}%
}

\providecommand{\rvCoords}{%
	\def\N{8}%
	\foreach \x/\lab in {1/A,2/B,3/C,4/D,5/E,6/F,7/G,8/H}{%
		\node[font=\rvCoordFont] at (\x-0.5,\N+0.65) {\lab};
	}%
	\foreach \r in {1,...,8}{%
		\node[font=\rvCoordFont] at (-0.65,\N-\r+0.5) {\r};
	}%
}

% ==========================================================
% Internal primitives (indices 0..7)
% Stones are drawn in cell units (like \rvMoveRad).
% ==========================================================
\providecommand{\rvStoneWhite}[2]{%
	\rvSetupScale%
	\path[draw=black, fill=white, line width=\rvStoneLW]
	(#1+0.5,#2+0.5) circle (\rvStoneRad);
}
\providecommand{\rvStoneBlack}[2]{%
	\rvSetupScale%
	\path[draw=black, fill=black, line width=\rvStoneLW]
	(#1+0.5,#2+0.5) circle (\rvStoneRad);
}

\providecommand{\rvMoveWhite}[2]{%
	\rvSetupScale%
	\draw[dashed, draw=black, line width=\rvMoveLW]
	(#1+0.5,#2+0.5) circle (\rvMoveRad);
}
\providecommand{\rvMoveBlack}[2]{%
	\rvSetupScale%
	\draw[draw=black, line width=\rvMoveLW]
	(#1+0.5,#2+0.5) circle (\rvMoveRad);
}

\providecommand{\rvMarkFrame}[2]{%
	\rvSetupScale%
	\draw[draw=red, line width=\rvMarkLW]
	(#1,#2) rectangle ++(1,1);
}

\providecommand{\rvLastMoveDot}[2]{%
	\rvSetupScale%
	\draw[draw=yellow!70!orange, fill=yellow!80!orange, line width=\rvLastMoveLW]
	(#1+0.5,#2+0.5) circle (\rvLastMoveRad);
}

\providecommand{\rvFlipSymbol}[2]{%
	\rvSetupScale%
	\draw[draw=yellow!70!orange, line width=\rvFlipLineLW, line cap=round]
	(#1+0.5, #2+\rvFlipYmin) -- (#1+0.5, #2+\rvFlipYmax);
	\path[draw=yellow!70!orange, fill=yellow!70!orange]
	(#1+0.5, #2+\rvFlipYmax+\rvFlipTriH) --
	(#1+0.5-\rvFlipTriW, #2+\rvFlipYmax) --
	(#1+0.5+\rvFlipTriW, #2+\rvFlipYmax) -- cycle;
	\path[draw=yellow!70!orange, fill=yellow!70!orange]
	(#1+0.5, #2+\rvFlipYmin-\rvFlipTriH) --
	(#1+0.5-\rvFlipTriW, #2+\rvFlipYmin) --
	(#1+0.5+\rvFlipTriW, #2+\rvFlipYmin) -- cycle;
}

\providecommand{\rvValueLabel}[3]{%
	\node[scale=\rvValueScale] at (#1+0.5,#2+0.5) {#3};
}

% ==========================================================
% Mapping: file/rank -> internal indices
% ==========================================================
\providecommand{\rvFileToX}[1]{%
	\ifnum\pdfstrcmp{#1}{A}=0 0\else
	\ifnum\pdfstrcmp{#1}{B}=0 1\else
	\ifnum\pdfstrcmp{#1}{C}=0 2\else
	\ifnum\pdfstrcmp{#1}{D}=0 3\else
	\ifnum\pdfstrcmp{#1}{E}=0 4\else
	\ifnum\pdfstrcmp{#1}{F}=0 5\else
	\ifnum\pdfstrcmp{#1}{G}=0 6\else
	\ifnum\pdfstrcmp{#1}{H}=0 7\else
	\ifnum\pdfstrcmp{#1}{a}=0 0\else
	\ifnum\pdfstrcmp{#1}{b}=0 1\else
	\ifnum\pdfstrcmp{#1}{c}=0 2\else
	\ifnum\pdfstrcmp{#1}{d}=0 3\else
	\ifnum\pdfstrcmp{#1}{e}=0 4\else
	\ifnum\pdfstrcmp{#1}{f}=0 5\else
	\ifnum\pdfstrcmp{#1}{g}=0 6\else
	\ifnum\pdfstrcmp{#1}{h}=0 7\else
	-1%
	\fi\fi\fi\fi\fi\fi\fi\fi
	\fi\fi\fi\fi\fi\fi\fi\fi
}
\providecommand{\rvRankToY}[1]{\numexpr8-#1\relax}

% ==========================================================
% User-facing API
% ==========================================================
\providecommand{\rvStoneBlackAt}[2]{\rvStoneBlack{\rvFileToX{#1}}{\rvRankToY{#2}}}
\providecommand{\rvStoneWhiteAt}[2]{\rvStoneWhite{\rvFileToX{#1}}{\rvRankToY{#2}}}
\providecommand{\rvMoveBlackAt}[2]{\rvMoveBlack{\rvFileToX{#1}}{\rvRankToY{#2}}}
\providecommand{\rvMoveWhiteAt}[2]{\rvMoveWhite{\rvFileToX{#1}}{\rvRankToY{#2}}}
\providecommand{\rvMarkFrameAt}[2]{\rvMarkFrame{\rvFileToX{#1}}{\rvRankToY{#2}}}

\providecommand{\rvStonesBlack}[1]{\foreach \p in {#1}{\expandafter\rvStoneBlackAux\p\relax}}
\providecommand{\rvStonesWhite}[1]{\foreach \p in {#1}{\expandafter\rvStoneWhiteAux\p\relax}}
\providecommand{\rvMovesBlack}[1]{\foreach \p in {#1}{\expandafter\rvMoveBlackAux\p\relax}}
\providecommand{\rvMovesWhite}[1]{\foreach \p in {#1}{\expandafter\rvMoveWhiteAux\p\relax}}
\providecommand{\rvMarkFrames}[1]{\foreach \p in {#1}{\expandafter\rvMarkFrameAux\p\relax}}
\providecommand{\rvFlips}[1]{\foreach \p in {#1}{\expandafter\rvFlipAux\p\relax}}

\providecommand{\rvLastMove}[1]{\expandafter\rvLastMoveAux#1\relax}
\def\rvLastMoveAux#1#2\relax{%
	\rvLastMoveDot{\rvFileToX{#1}}{\rvRankToY{#2}}%
}

\providecommand{\rvValueMap}[1]{\foreach \pv in {#1}{\expandafter\rvValueMapAux\pv\relax}}
\def\rvValueMapAux#1:#2\relax{%
	\expandafter\rvValueMapCoordAux#1\relax{#2}%
}
\def\rvValueMapCoordAux#1#2\relax#3{%
	\rvValueLabel{\rvFileToX{#1}}{\rvRankToY{#2}}{#3}%
}

% Coordinate parser for tokens like "E4"
\def\rvStoneBlackAux#1#2\relax{\rvStoneBlackAt{#1}{#2}}
\def\rvStoneWhiteAux#1#2\relax{\rvStoneWhiteAt{#1}{#2}}
\def\rvMoveBlackAux#1#2\relax{\rvMoveBlackAt{#1}{#2}}
\def\rvMoveWhiteAux#1#2\relax{\rvMoveWhiteAt{#1}{#2}}
\def\rvMarkFrameAux#1#2\relax{\rvMarkFrameAt{#1}{#2}}
\def\rvFlipAux#1#2\relax{\rvFlipSymbol{\rvFileToX{#1}}{\rvRankToY{#2}}}

% Matrix input to display values on board
\providecommand{\rvValueMatrix}[8]{%
	\rvValueMatrixRow{#1}{7}%
	\rvValueMatrixRow{#2}{6}%
	\rvValueMatrixRow{#3}{5}%
	\rvValueMatrixRow{#4}{4}%
	\rvValueMatrixRow{#5}{3}%
	\rvValueMatrixRow{#6}{2}%
	\rvValueMatrixRow{#7}{1}%
	\rvValueMatrixRow{#8}{0}%
}

\def\rvValueMatrixRow#1#2{%
	\foreach \v [count=\x from 0] in {#1}{%
		\rvValueLabel{\x}{#2}{\v}%
	}%
}

		\subfigure[Spielsituation vor dem Zug]{
			\begin{tikzpicture}[scale=0.6]
				\rvBoard
				\rvCoords
				\rvStonesBlack{A5, D4}
				\rvStonesWhite{A3, A4, B4, C4}
				\rvMovesBlack{A2}
			\end{tikzpicture}
		}
		\hspace{12mm}
		\subfigure[Spielsituation nach dem Zug]{
			\begin{tikzpicture}[scale=0.6]
				\rvBoard
				\rvCoords
				\rvStonesBlack{A5, D4, A2, A3, A4}
				\rvStonesWhite{B4, C4}
				\rvFlips{A3, A4}
				\rvLastMove{A2}
			\end{tikzpicture}
		}
		\caption[Spielzug mit Überflügelung nur in direkter Linie]{Spielzug mit Überflügelung nur in direkter Linie vor (a) und nach dem Zug (b).}
		\label{fig:othello-ueberfluegeln-in-direkter-linie}
	\end{figure}
	\item Alle in einem Zug überflügelten Spielsteine müssen umgedreht werden, auch wenn es für den Spieler vorteilhaft wäre, sie nicht umzudrehen.
	
	\item Sobald ein Spielstein auf einem Feld platziert wurde, darf er im weiteren Spielverlauf nicht mehr auf ein anderes Feld bewegt werden.
	
	\item Kann keiner der beiden Spieler einen weiteren Zug ausführen, endet das Spiel. Anschließend werden die Spielsteine gezählt, und der Spieler mit der höheren Anzahl an Spielsteinen seiner Farbe gewinnt.
\end{enumerate}

\section{Spieltheoretische Grundlagen adversarieller Suche}
\label{sec:spieltheorie}

Die Spieltheorie stellt einen formalen Rahmen zur Beschreibung und Analyse von interaktiven, strategischen Situationen bereit, in denen mehrere Akteure, sogenannte \emph{Spieler}, Handlungen ausführen, die sich gegenseitig beeinflussen. Sie findet Anwendung in unterschiedlichen Fachdisziplinen, wobei der Begriff des Spielers jeweils unterschiedlich interpretiert wird, etwa als lebender Organismus in der Evolutionsbiologie, als wirtschaftlicher Akteur in der Ökonomie oder als künstlicher Agent in der Informatik \autocite[Kap.~1]{bonannoGameTheory2024}.

In dieser Arbeit wird die Spieltheorie zur formalen Beschreibung und Analyse des Brettspiels Othello herangezogen, um darauf aufbauend geeignete Spielalgorithmen zu entwerfen. Ziel des folgenden Kapitels ist es, die spieltheoretischen Grundlagen zu schaffen, die für den Vergleich verschiedener Spielalgorithmen in Kapitel~\ref{cha:stand-der-technik} erforderlich sind.

Die Betrachtung beschränkt sich bewusst auf die adversarielle Suche. Dabei handelt es sich um eine Klasse von Suchverfahren der künstlichen Intelligenz für Mehrspielerprobleme mit gegensätzlichen Zielen, wie sie bei Othello vorliegen. Diese Verfahren ermöglichen eine strukturierte und deterministische Entscheidungsfindung. Zudem eignen sie sich besonders für Systeme mit begrenzten Rechenressourcen, wie sie bei der Hardware des LEGO\textsuperscript{\textregistered} SPIKE Systems gegeben sind.

In Abschnitt~\ref{subsec:modellierung-spiele-baum} werden grundlegende Begriffe der Spieltheorie eingeführt und Spiele formal als Suchbäume modelliert. Darauf aufbauend behandelt Abschnitt~\ref{subsec:prinzipien-kompetitive-spiele} zentrale Prinzipien der Entscheidungsfindung in kompetitiven Spielen, die im Wesentlichen adversariale Suchprobleme darstellen. Abschließend werden die vorgestellten Konzepte in Abschnitt~\ref{subsec:spieltheoretische-einordnung-othello} auf das Spiel Othello angewandt und dieses aus spieltheoretischer Sicht eingeordnet.

\subsection{Formale Modellierung von Spielen als Suchbäume}
\label{subsec:modellierung-spiele-baum}
Die Spieltheorie lässt sich in zwei Hauptbereiche unterteilen: die kooperative und die nicht-kooperative Spieltheorie. Die kooperative Spieltheorie geht davon aus, dass Spieler miteinander kommunizieren, Koalitionen bilden und bindende Vereinbarungen treffen können. Für diese Arbeit ist die nicht-kooperative, auch kompetitive Spieltheorie relevant, weshalb sich die folgende Betrachtung auf diesen Bereich beschränkt. In diesem Kontext können Spieler entweder nicht miteinander kommunizieren oder sind zwar zur Kommunikation fähig, können jedoch keine bindenden Vereinbarungen treffen. \autocite[Kap.~1]{bonannoGameTheory2024}

Im Folgenden werden die grundlegenden Begriffe der Spieltheorie anhand von Beispielen eingeführt und definiert. Der Einstieg erfolgt über ein simultan verlaufendes Standardbeispiel aus der britischen TV-Show \emph{Golden Balls}, da sich zentrale Konzepte der Spieltheorie in diesem Kontext anschaulich erläutern lassen. Anschließend werden die eingeführten Begriffe auf das Spiel Othello übertragen und um für sequentielle Spiele relevante Aspekte ergänzt.

Das Spiel \emph{Golden Balls} funktioniert wie folgt: Zwei Spieler, Anton und Berta, entscheiden gleichzeitig und unabhängig voneinander, ob sie kooperieren oder defektieren. Wählen beide Spieler Kooperation, wird der Gewinn, beispielsweise in Höhe von 100\,€, gleichmäßig aufgeteilt. Entscheidet sich ein Spieler für Defektion, während der andere kooperiert, erhält der defektierende Spieler den gesamten Gewinn. Defektieren beide Spieler, erhalten beide keinen Gewinn. Die möglichen Entscheidungsausgänge sind in Tabelle~\ref{tab:golden-balls} zusammengefasst.

\begin{table}[hbt]	
	\centering
	\captionabove[Ausgänge des Spiels \emph{Golden Balls}]{Mögliche Ausgänge des Spiels \emph{Golden Balls}. (Basierend auf \autocite[Abb.~2.1]{bonannoGameTheory2024})}
	\label{tab:golden-balls}
	
	\renewcommand{\arraystretch}{1.5}
	
	\begin{tabular}{c c|c c c c}
		 & \multicolumn{1}{c}{} & \multicolumn{4}{c}{\large \textcolor{blue}{Anton \normalsize($i=1$)}} \\
		 & & \multicolumn{2}{c|}{kooperiert ($s_{11}$)} & \multicolumn{2}{c|}{defektiert ($s_{12}$)}\\ \cline{2-6}
		\multirow{4}{*}{\large \textcolor{red}{\shortstack{Berta\\ \normalsize($i=2$)}}} & \multirow{2}{*}{kooperiert ($s_{21}$)} & \multirow{2}{*}{$o_1:$} & \multicolumn{1}{l|}{\textcolor{blue}{Anton erhält 50\,€,}} & \multirow{2}{*}{$o_3:$} & \multicolumn{1}{l|}{\textcolor{blue}{Anton erhält 100\,€,}} \\
		 & & & \multicolumn{1}{l|}{\textcolor{red}{Berta erhält 50\,€.}} & & \multicolumn{1}{l|}{\textcolor{red}{Berta erhält 0\,€.}} \\ \cline{2-6}
		 & \multirow{2}{*}{defektiert ($s_{22}$)} & \multirow{2}{*}{$o_2:$} & \multicolumn{1}{l|}{\textcolor{blue}{Anton erhält 0\,€,}} & \multirow{2}{*}{$o_4:$} & \multicolumn{1}{l|}{\textcolor{blue}{Anton erhält 0\,€,}}  \\
		 & & & \multicolumn{1}{l|}{\textcolor{red}{Berta erhält 100\,€.}} & & \multicolumn{1}{l|}{\textcolor{red}{Berta erhält 0\,€.}}\\ \cline{2-6}
	\end{tabular}
	
\end{table}

Formal lässt sich die \emph{Spielsituation} wie folgt beschreiben: Eine \emph{Menge von Spielern}
$ I = \{1, 2\}$, wobei $1=\text{Anton}$ und $2=\text{Berta}$, verfügt jeweils über zwei \emph{reine Strategien}
$ s_i $, nämlich \glqq Kooperieren\grqq{} ($ s_{i1} $) oder \glqq Defektieren\grqq{} ($ s_{i2} $).
Abhängig von der Wahl der Strategien ergibt sich eine \emph{Strategiekombination} $ s $, die als
geordnetes Paar, etwa $ S_i=(s_{i1}, s_{i2}) $, dargestellt wird.
Der \emph{Strategieraum}
$ S = S_1 \times S_2 = \{s_1, s_2, s_3, s_4\} = \{(s_{11}, s_{21}), (s_{11}, s_{22}), (s_{12}, s_{21}), (s_{12}, s_{22  })\} $ umfasst die Menge aller möglichen
Kombinationen der Strategien $ s_i \in S_i $. Insgesamt existieren somit vier ($2 \times 2$)
mögliche Kombinationen reiner Strategien.
Jede Strategiekombination $ s $ bestimmt ein \emph{Ergebnis} $ o $. Die Menge aller möglichen
Ergebnisse wird als \emph{Ergebnisraum}
$ O = \{o_1, o_2, o_3, o_4\} $ bezeichnet.
Die Darstellung der Spielsituation in der Form $ \Gamma^{\dagger} = \langle I, S, O \rangle  $ wird als \emph{Spielform}
bezeichnet und lässt sich beispielsweise in Matrixform, wie in
Tabelle~\ref{tab:golden-balls}, darstellen. Wird diese um eine \emph{Ergebnisfunktion}
$ f : S \to O $ mit $ f(s) = o_i \in O $ ergänzt, die jeder Strategiekombination $ s $
ein Ergebnis zuweist, ergibt sich die Darstellung $ \Gamma^{\dagger \dagger} = \langle I, S, O, f \rangle $, die auch als
\emph{Game Frame} bezeichnet wird. \autocite[Kap.~2.1]{bonannoGameTheory2024}, \autocite[Kap.~1.2.1]{hollerEinfuehrungSpieltheorie2019}

Das Spiel ist in diesem Stadium formal noch unvollständig beschrieben, da keine Priorisierung der Ergebnisse durch die Spieler vorliegt. Um eine Entscheidung treffen zu können, müssen die möglichen Entscheidungsausgänge bewertet werden. Dies erfordert neben der Kenntnis aller Ergebnisse auch deren relative Bewertung zueinander. Formal ausgedrückt müssen die Ausgänge in eine vollständige und transitive Ordnung überführt werden.
Beispielhaft kann angenommen werden, dass beide Spieler egoistisch und nutzenmaximierend handeln.  In diesem Fall ergibt sich für jeden Spieler eine individuelle Präferenzordnung über die möglichen Entscheidungsausgänge. Für Spieler~1 (Anton) ergibt sich die Präferenzordnung
$ o_3 \succ_1 o_1 \succ_1 o_2 \sim_1 o_4 $,
für Spieler~2 (Berta) entsprechend die Präferenzordnung
$ o_2 \succ_2 o_1 \succ_2 o_3 \sim_2 o_4 $.
(Anmerkung zur Notation:
	$o \succeq_i o'$ bezeichnet eine \emph{schwache Präferenz};
	$o \succ_i o'$ eine \emph{strikte Präferenz};
	$o \sim_i o'$ \emph{Indifferenz}.)
Die Annahme egoistischen und nutzenmaximierenden Handelns ist dabei frei gewählt und unter alternativen Annahmen ergeben sich entsprechend andere Präferenzordnungen.
\autocite[Kap.~2.1]{bonannoGameTheory2024}, \autocite[Kap.~1.1]{bartholomaeSpieltheorie2016}

Auf Grundlage der zuvor eingeführten Konzepte kann die Definition des Game Frames erweitert werden. Daraus ergibt sich gemäß \autocite[Def.~2.1.3]{bonannoGameTheory2024} die nachfolgende Definition~\ref{def:ordinales-spiel} eines \emph{ordinalen Spiels in strategischer Form}.
\begin{definition}
	\label{def:ordinales-spiel}
	Ein \emph{ordinales Spiel in strategischer Form} ist ein Quintupel
	\[
	\Gamma=\langle I,(S_i)_{i\in I},O,f,(\succeq_i)_{i\in I}\rangle,
	\quad \text{wobei}
	\]
	\begin{itemize}[
		itemsep=0pt,
		topsep=0pt,
		parsep=0pt,
		partopsep=0pt
		]
		\item $I=\{1,2,\ldots,n\}$ eine endliche Menge von Spielern mit $n\geq 2$ ist,
		\item $S_i$ die Menge der Strategien von Spieler $i\in I$ bezeichnet und
		$S=S_1 \times S_2 \times \ldots \times S_n$ der Strategieraum ist,
		\item $O$ die Menge der möglichen Ergebnisse ist,
		\item $f:S \to O$ eine Ergebnisfunktion ist, die jedem
		Strategienprofil $s\in S$ genau ein Ergebnis $f(s)\in O$ zuordnet,
		\item $\succeq_i$ eine vollständige und transitive Präferenzrelation von
		Spieler $i\in I$ über der Ergebnismenge $O$ ist.
	\end{itemize}
\end{definition}

Neben der Darstellung von Präferenzen durch eine Ordnung können diese auch mittels einer \emph{Nutzenfunktion} dargestellt werden. Dabei wird jedem möglichen Ergebnis ein reeller Zahlenwert zugeordnet, wobei ein Ergebnis \(o\) gegenüber einem Ergebnis \(o'\) präferiert wird, wenn der zugehörige Nutzenwert größer ist. Formal ist die Nutzenfunktion als Abbildung \(U \colon O \to \mathbb{R}\) definiert, die die Rangfolge der Präferenzen eines Spielers widerspiegelt. Ein höherer Nutzenwert entspricht einem aus Sicht des Spielers besseren Ergebnis. Nutzenwerte besitzen jedoch keine kardinale Bedeutung, sondern sind ordinal zu interpretieren. Die Wahl der Zahlenwerte ist daher grundsätzlich arbiträr; ein doppelt so hoher Nutzenwert impliziert kein doppelt so gutes Ergebnis. Unter der Annahme egoistischen und nutzenmaximierenden Handelns kann für jeden Spieler eine entsprechende Nutzenfunktion festgelegt werden, wie sie beispielhaft in Tabelle~\ref{tab:nutzenfunktion-golden-balls} dargestellt ist. \autocite[Kap.~2.1]{bonannoGameTheory2024}

\begin{table}[hbt]
	\centering
	\captionabove[Nutzenfunktion der Spieler im Spiel Golden Balls]{Nutzenfunktion der Spieler im Spiel Golden Balls unter der Annahme egoistischen und nutzenmaximierenden Handelns.}
	\label{tab:nutzenfunktion-golden-balls}
	
	\renewcommand{\arraystretch}{1.5}
	
	\begin{tabular}{c|c c c c}
		Ergebnis $\rightarrow$ & $o_1$ & $o_2$ & $o_3$ & $o_4$ \\
		Nutzenfunktion $\downarrow$ & & & & \\ \hline
		\textcolor{blue}{$U_1$ (Anton):} & \textcolor{blue}{$3$} & \textcolor{blue}{$2$} & \textcolor{blue}{$4$} & \textcolor{blue}{$2$} \\
		\textcolor{red}{$U_2$ (Berta):} & \textcolor{red}{$3$} & \textcolor{red}{$4$} & \textcolor{red}{$2$} & \textcolor{red}{$2$} \\
	\end{tabular}
\end{table}

Basierend auf der Repräsentation der Präferenzen durch Nutzenfunktionen lässt sich ein \emph{ordinales Spiel in reduzierter Form} definieren. Diese in der Literatur verbreitete Darstellung verzichtet auf eine explizite Spezifikation der Ergebnismenge und wird in der folgenden Definition formalisiert (vgl.~\autocite[Def.~2.1.3]{bonannoGameTheory2024}).

\begin{definition}
	\label{def:reduziertes-ordinales-spiel}
	Ein \emph{reduziertes ordinales Spiel in strategischer Form} ist ein Tripel
	\[
	\Gamma^{\prime}=\langle I,(S_i)_{i\in I},(\pi_i)_{i\in I}\rangle,
	\quad \text{wobei}
	\]
	\begin{itemize}[
		itemsep=0pt,
		topsep=0pt,
		parsep=0pt,
		partopsep=0pt
		]
		\item $I=\{1,2,\ldots,n\}$ eine endliche Menge von Spielern mit $n\geq 2$ ist,
		\item $S_i$ die Menge der Strategien von Spieler $i\in I$ bezeichnet,
		\item $\pi_i : S \to \mathbb{R}$ die Auszahlungsfunktion des Spiels $i$ ist, d.h. $\pi_i=U_i(f(s))$.
	\end{itemize}
\end{definition}

Im Gegensatz zu simultanen Interaktionen verlaufen viele Spiele sequentiell. Dies gilt auch für Othello, bei dem die Spieler Schwarz und Weiß abwechselnd ziehen und Kenntnis aller bisherigen Züge besitzen. Solche Spiele werden als \emph{dynamische Spiele} oder \emph{Spiele in extensiver Form} bezeichnet. Othello zählt zudem zu den Spielen mit \emph{perfekter Information}, da jedem Spieler zum Zeitpunkt seines Zuges alle vorherigen Züge bekannt sind.

Spiele dieser Art lassen sich formal in strategischer Form (Normalform) darstellen, sodass die zuvor eingeführten Definitionen ihre Gültigkeit behalten. Für praktische Analysen ist diese Darstellung jedoch ungeeignet, da Strategien und einzelne Entscheidungen, anders als im Beispiel \emph{Golden Balls}, nicht übereinstimmen. In dynamischen Spielen mit perfekter Information ist eine Strategie als vollständiger Handlungsplan zu verstehen, der für jede mögliche Spielsituation eine zulässige Aktion festlegt. Der Strategieraum $S = S_1 \times S_2$ umfasst daher alle Kombinationen solcher Pläne und ist entsprechend komplex. Eine explizite Auflistung der Strategiemengen oder eine Darstellung in Matrixform ist folglich nicht praktikabel. Für die Analyse und die algorithmische Umsetzung sequentieller Spiele wird stattdessen die extensive Form verwendet, in der das Spiel als Baum dargestellt wird. Im Folgenden wird die Definition solcher gewurzelten, gerichteten Spielbäume eingeführt.

\begin{definition}
	\label{def:rooted-directed-tree}
	Ein \emph{gewurzelter gerichteter Baum} besteht aus einer Menge von Knoten und gerichteten Kanten, die diese verbinden. (vgl.~\autocite[Def.~3.1.1]{bonannoGameTheory2024})
	\begin{itemize}[
		itemsep=0pt,
		topsep=0pt,
		parsep=0pt,
		partopsep=0pt,
		before=\vspace{-0.5\baselineskip}
		]
		\item Die Wurzel besitzt keinen eingehenden Pfeil (Eingangsgrad $0$), während jeder andere Knoten genau einen eingehenden Pfeil besitzt (Eingangsgrad $1$).
		\item Von der Wurzel zu jedem anderen Knoten existiert genau ein eindeutiger Pfad, das heißt eine eindeutige Folge gerichteter Kanten.
		\item Ein Knoten ohne ausgehende Pfeile (Ausgangsgrad $0$) heißt Endknoten, jeder andere Knoten heißt Entscheidungsknoten.
		\item Die Menge aller Knoten wird mit $X$ bezeichnet, die Menge der Entscheidungsknoten mit $D$ und die Menge der Endknoten mit $Z$, sodass $X = D \cup Z$ gilt.
	\end{itemize}
\end{definition}

\begin{definition}
	\label{def:finite-extensive-form}
	Eine \emph{endliche extensive Form mit perfekter Information} (auch \emph{extensive-form game frame}) besteht aus den folgenden Bestandteilen (vgl.~\autocite[Def.~3.1.2]{bonannoGameTheory2024}):
	\begin{itemize}[
		itemsep=0pt,
		topsep=0pt,
		parsep=0pt,
		partopsep=0pt,
		before=\vspace{-0.5\baselineskip}
		]
		\item einem endlichen gewurzelten gerichteten Baum,
		\item einer Spielermenge $I=\{1,\ldots,n\}$ und einer Zuordnung, die jedem Entscheidungsknoten genau einen Spieler zuweist,
		\item einer Aktionsmenge $A$ und einer Zuordnung, die jeder gerichteten Kante genau eine Aktion zuweist, wobei aus demselben Knoten keine zwei Kanten dieselbe Aktion tragen,
		\item einer Ergebnismenge $O$ und einer Zuordnung, die jedem Endknoten genau ein Ergebnis zuweist.
	\end{itemize}
\end{definition}

\begin{definition}
	\label{def:finite-extensive-game}
	Ein \emph{endliches extensives Spiel mit perfekter Information} ist eine endliche extensive Form mit perfekter Information zusammen mit einer Rangordnung der Ergebnismenge.
	Für jeden Spieler $i \in I$ ist eine Präferenzrelation $\succeq_i$ auf $O$ gegeben, die üblicherweise durch eine ordinale Nutzenfunktion $U_i : O \to \mathbb{R}$ dargestellt werden kann (vgl.~\autocite[Def.~3.1.3]{bonannoGameTheory2024}).
\end{definition}

Zur Veranschaulichung dieser eingeführten Definitionen werden die initialen Züge des Spiels Othello betrachtet und ein entsprechender Spielbaum konstruiert.

Abbildung~\ref{fig:othello-initiale-spielsituationen} zeigt die ersten möglichen Züge von Schwarz und Weiß.
In der Unterabbildung~(i) eröffnet Schwarz das Spiel ausgehend von der Startaufstellung; dargestellt sind die zulässigen Anfangszüge. In den Unterabbildungen~(a) bis~(d) ist Weiß am Zug. Gezeigt werden jeweils die resultierenden Spielsituationen in Abhängigkeit vom zuvor ausgeführten Zug von Schwarz sowie die daraus folgenden möglichen Aktionen von Weiß.
Es sei angemerkt, dass aufgrund der Symmetrieeigenschaften von Othello einzelne Spielsituationen zusammengefasst werden könnten. Auf eine solche Reduktion wird jedoch bewusst verzichtet, um im Folgenden den vollständigen Spielbaum darzustellen.

\begin{figure}[hbt]
	\centering
	% tikz/reversi-board.tex
% Reversi/Othello board (8x8) as reusable TikZ macros (no tikzpicture).
% Convention: A1 is top-left. Internal indices: A1 -> (0,7).

% ==========================================================
% Parameters (cell units / cell fractions)
% - Anything used in "circle(...)" is in cell units (like \rvMoveRad).
% - Line widths and node sizes need TeX lengths -> derived from cell length.
% ==========================================================
\providecommand{\rvBoardColor}{green!35}

% Geometry (CELL UNITS)
% Stones are drawn like moves: radius in cell units.
\providecommand{\rvStoneRad}{0.4}      % 0.50 => diameter exactly 1 cell
\providecommand{\rvMoveRad}{0.4}       % move marker radius (cell units)
\providecommand{\rvLastMoveRad}{0.1}   % last-move dot radius (cell units)

% Line widths (CELL FRACTIONS of one cell)
\providecommand{\rvOuterLWFrac}{0.2}
\providecommand{\rvGridLWFrac}{0.1}
\providecommand{\rvStoneLWFrac}{0.4}
\providecommand{\rvMoveLWFrac}{0.4}
\providecommand{\rvLastMoveLWFrac}{0.4}
\providecommand{\rvMarkLWFrac}{1.0}

% Flip marker (CELL UNITS + line width as fraction)
\providecommand{\rvFlipLineLWFrac}{1.0}
\providecommand{\rvFlipYmin}{0.38}
\providecommand{\rvFlipYmax}{0.62}
\providecommand{\rvFlipTriH}{0.13}
\providecommand{\rvFlipTriW}{0.12}

% Text
\providecommand{\rvCoordFont}{\small}
\providecommand{\rvValueScale}{0.95}

% ==========================================================
% Internal lengths (defined once even if file is input multiple times)
% ==========================================================
\makeatletter
\@ifundefined{rvCellLen}{\newlength{\rvCellLen}}{}
\@ifundefined{rvOuterLW}{\newlength{\rvOuterLW}}{}
\@ifundefined{rvGridLW}{\newlength{\rvGridLW}}{}
\@ifundefined{rvStoneLW}{\newlength{\rvStoneLW}}{}
\@ifundefined{rvMoveLW}{\newlength{\rvMoveLW}}{}
\@ifundefined{rvLastMoveLW}{\newlength{\rvLastMoveLW}}{}
\@ifundefined{rvMarkLW}{\newlength{\rvMarkLW}}{}
\@ifundefined{rvFlipLineLW}{\newlength{\rvFlipLineLW}}{}
\makeatother

% ==========================================================
% Scale derivation (recomputed when drawing a board)
% ==========================================================
\providecommand{\rvSetupScale}{%
	\pgfextractx{\rvCellLen}{\pgfpoint{1}{0}}%
	\pgfmathsetlength{\rvOuterLW}{\rvOuterLWFrac*\rvCellLen}%
	\pgfmathsetlength{\rvGridLW}{\rvGridLWFrac*\rvCellLen}%
	\pgfmathsetlength{\rvStoneLW}{\rvStoneLWFrac*\rvCellLen}%
	\pgfmathsetlength{\rvMoveLW}{\rvMoveLWFrac*\rvCellLen}%
	\pgfmathsetlength{\rvLastMoveLW}{\rvLastMoveLWFrac*\rvCellLen}%
	\pgfmathsetlength{\rvMarkLW}{\rvMarkLWFrac*\rvCellLen}%
	\pgfmathsetlength{\rvFlipLineLW}{\rvFlipLineLWFrac*\rvCellLen}%
}

% ==========================================================
% Board
% ==========================================================
\providecommand{\rvBoard}{%
	\rvSetupScale%
	\def\N{8}%
	\fill[\rvBoardColor] (0,0) rectangle (\N,\N);
	\draw[line width=\rvOuterLW] (0,0) rectangle (\N,\N);
	\foreach \i in {1,...,7}{%
		\draw[line width=\rvGridLW] (\i,0) -- (\i,\N);
		\draw[line width=\rvGridLW] (0,\i) -- (\N,\i);
	}%
}

\providecommand{\rvCoords}{%
	\def\N{8}%
	\foreach \x/\lab in {1/A,2/B,3/C,4/D,5/E,6/F,7/G,8/H}{%
		\node[font=\rvCoordFont] at (\x-0.5,\N+0.65) {\lab};
	}%
	\foreach \r in {1,...,8}{%
		\node[font=\rvCoordFont] at (-0.65,\N-\r+0.5) {\r};
	}%
}

% ==========================================================
% Internal primitives (indices 0..7)
% Stones are drawn in cell units (like \rvMoveRad).
% ==========================================================
\providecommand{\rvStoneWhite}[2]{%
	\rvSetupScale%
	\path[draw=black, fill=white, line width=\rvStoneLW]
	(#1+0.5,#2+0.5) circle (\rvStoneRad);
}
\providecommand{\rvStoneBlack}[2]{%
	\rvSetupScale%
	\path[draw=black, fill=black, line width=\rvStoneLW]
	(#1+0.5,#2+0.5) circle (\rvStoneRad);
}

\providecommand{\rvMoveWhite}[2]{%
	\rvSetupScale%
	\draw[dashed, draw=black, line width=\rvMoveLW]
	(#1+0.5,#2+0.5) circle (\rvMoveRad);
}
\providecommand{\rvMoveBlack}[2]{%
	\rvSetupScale%
	\draw[draw=black, line width=\rvMoveLW]
	(#1+0.5,#2+0.5) circle (\rvMoveRad);
}

\providecommand{\rvMarkFrame}[2]{%
	\rvSetupScale%
	\draw[draw=red, line width=\rvMarkLW]
	(#1,#2) rectangle ++(1,1);
}

\providecommand{\rvLastMoveDot}[2]{%
	\rvSetupScale%
	\draw[draw=yellow!70!orange, fill=yellow!80!orange, line width=\rvLastMoveLW]
	(#1+0.5,#2+0.5) circle (\rvLastMoveRad);
}

\providecommand{\rvFlipSymbol}[2]{%
	\rvSetupScale%
	\draw[draw=yellow!70!orange, line width=\rvFlipLineLW, line cap=round]
	(#1+0.5, #2+\rvFlipYmin) -- (#1+0.5, #2+\rvFlipYmax);
	\path[draw=yellow!70!orange, fill=yellow!70!orange]
	(#1+0.5, #2+\rvFlipYmax+\rvFlipTriH) --
	(#1+0.5-\rvFlipTriW, #2+\rvFlipYmax) --
	(#1+0.5+\rvFlipTriW, #2+\rvFlipYmax) -- cycle;
	\path[draw=yellow!70!orange, fill=yellow!70!orange]
	(#1+0.5, #2+\rvFlipYmin-\rvFlipTriH) --
	(#1+0.5-\rvFlipTriW, #2+\rvFlipYmin) --
	(#1+0.5+\rvFlipTriW, #2+\rvFlipYmin) -- cycle;
}

\providecommand{\rvValueLabel}[3]{%
	\node[scale=\rvValueScale] at (#1+0.5,#2+0.5) {#3};
}

% ==========================================================
% Mapping: file/rank -> internal indices
% ==========================================================
\providecommand{\rvFileToX}[1]{%
	\ifnum\pdfstrcmp{#1}{A}=0 0\else
	\ifnum\pdfstrcmp{#1}{B}=0 1\else
	\ifnum\pdfstrcmp{#1}{C}=0 2\else
	\ifnum\pdfstrcmp{#1}{D}=0 3\else
	\ifnum\pdfstrcmp{#1}{E}=0 4\else
	\ifnum\pdfstrcmp{#1}{F}=0 5\else
	\ifnum\pdfstrcmp{#1}{G}=0 6\else
	\ifnum\pdfstrcmp{#1}{H}=0 7\else
	\ifnum\pdfstrcmp{#1}{a}=0 0\else
	\ifnum\pdfstrcmp{#1}{b}=0 1\else
	\ifnum\pdfstrcmp{#1}{c}=0 2\else
	\ifnum\pdfstrcmp{#1}{d}=0 3\else
	\ifnum\pdfstrcmp{#1}{e}=0 4\else
	\ifnum\pdfstrcmp{#1}{f}=0 5\else
	\ifnum\pdfstrcmp{#1}{g}=0 6\else
	\ifnum\pdfstrcmp{#1}{h}=0 7\else
	-1%
	\fi\fi\fi\fi\fi\fi\fi\fi
	\fi\fi\fi\fi\fi\fi\fi\fi
}
\providecommand{\rvRankToY}[1]{\numexpr8-#1\relax}

% ==========================================================
% User-facing API
% ==========================================================
\providecommand{\rvStoneBlackAt}[2]{\rvStoneBlack{\rvFileToX{#1}}{\rvRankToY{#2}}}
\providecommand{\rvStoneWhiteAt}[2]{\rvStoneWhite{\rvFileToX{#1}}{\rvRankToY{#2}}}
\providecommand{\rvMoveBlackAt}[2]{\rvMoveBlack{\rvFileToX{#1}}{\rvRankToY{#2}}}
\providecommand{\rvMoveWhiteAt}[2]{\rvMoveWhite{\rvFileToX{#1}}{\rvRankToY{#2}}}
\providecommand{\rvMarkFrameAt}[2]{\rvMarkFrame{\rvFileToX{#1}}{\rvRankToY{#2}}}

\providecommand{\rvStonesBlack}[1]{\foreach \p in {#1}{\expandafter\rvStoneBlackAux\p\relax}}
\providecommand{\rvStonesWhite}[1]{\foreach \p in {#1}{\expandafter\rvStoneWhiteAux\p\relax}}
\providecommand{\rvMovesBlack}[1]{\foreach \p in {#1}{\expandafter\rvMoveBlackAux\p\relax}}
\providecommand{\rvMovesWhite}[1]{\foreach \p in {#1}{\expandafter\rvMoveWhiteAux\p\relax}}
\providecommand{\rvMarkFrames}[1]{\foreach \p in {#1}{\expandafter\rvMarkFrameAux\p\relax}}
\providecommand{\rvFlips}[1]{\foreach \p in {#1}{\expandafter\rvFlipAux\p\relax}}

\providecommand{\rvLastMove}[1]{\expandafter\rvLastMoveAux#1\relax}
\def\rvLastMoveAux#1#2\relax{%
	\rvLastMoveDot{\rvFileToX{#1}}{\rvRankToY{#2}}%
}

\providecommand{\rvValueMap}[1]{\foreach \pv in {#1}{\expandafter\rvValueMapAux\pv\relax}}
\def\rvValueMapAux#1:#2\relax{%
	\expandafter\rvValueMapCoordAux#1\relax{#2}%
}
\def\rvValueMapCoordAux#1#2\relax#3{%
	\rvValueLabel{\rvFileToX{#1}}{\rvRankToY{#2}}{#3}%
}

% Coordinate parser for tokens like "E4"
\def\rvStoneBlackAux#1#2\relax{\rvStoneBlackAt{#1}{#2}}
\def\rvStoneWhiteAux#1#2\relax{\rvStoneWhiteAt{#1}{#2}}
\def\rvMoveBlackAux#1#2\relax{\rvMoveBlackAt{#1}{#2}}
\def\rvMoveWhiteAux#1#2\relax{\rvMoveWhiteAt{#1}{#2}}
\def\rvMarkFrameAux#1#2\relax{\rvMarkFrameAt{#1}{#2}}
\def\rvFlipAux#1#2\relax{\rvFlipSymbol{\rvFileToX{#1}}{\rvRankToY{#2}}}

% Matrix input to display values on board
\providecommand{\rvValueMatrix}[8]{%
	\rvValueMatrixRow{#1}{7}%
	\rvValueMatrixRow{#2}{6}%
	\rvValueMatrixRow{#3}{5}%
	\rvValueMatrixRow{#4}{4}%
	\rvValueMatrixRow{#5}{3}%
	\rvValueMatrixRow{#6}{2}%
	\rvValueMatrixRow{#7}{1}%
	\rvValueMatrixRow{#8}{0}%
}

\def\rvValueMatrixRow#1#2{%
	\foreach \v [count=\x from 0] in {#1}{%
		\rvValueLabel{\x}{#2}{\v}%
	}%
}

	
	\leavevmode
	\makebox[\textwidth][c]{%
		% ------------------ LINKS: ------------------
		\begin{minipage}[c]{0.32\textwidth}
			\centering
			\vspace{12mm}
			\begin{tikzpicture}[scale=0.5]
				\rvBoard
				\rvCoords
				\rvStonesBlack{E4, D5}
				\rvStonesWhite{E5, D4}
				\rvMovesBlack{D3, C4, F5, E6}
				\rvValueMap{C4:a, D3:b, E6:c, F5:d}
			\end{tikzpicture}
			\vfill
			\par\vspace{0.4em}
			{\footnotesize (i) Startaufstellung}
		\end{minipage}%
		\hspace{0.04\textwidth}
		% ------------------ RECHTS: ------------------
		\begin{minipage}[t]{0.64\textwidth}
			\centering
			
			% ===== Reihe 1 =====
			\begin{minipage}[t]{0.49\linewidth}
				\centering
				\begin{tikzpicture}[scale=0.5]
					\rvBoard
					\rvCoords
					\rvStonesBlack{C4, D4, E4, D5}
					\rvStonesWhite{E5}
					\rvLastMove{C4}
					\rvFlips{D4}
					\rvMovesWhite{C3, E3, C5}
				\end{tikzpicture}
				\par\vspace{0.2em}{\footnotesize (a) Schwarz zuvor: C4}
			\end{minipage}%
			\hfill
			\begin{minipage}[t]{0.49\linewidth}
				\centering
				\begin{tikzpicture}[scale=0.5]
					\rvBoard
					\rvCoords
					\rvStonesBlack{E4, D5, D3, D4}
					\rvStonesWhite{E5}
					\rvLastMove{D3}
					\rvFlips{D4}
					\rvMovesWhite{C3, E3, C5}
				\end{tikzpicture}
				\par\vspace{0.2em}{\footnotesize (b) Schwarz zuvor: D3}
			\end{minipage}
			
			\vspace{4mm}
			
			% ===== Reihe 2 =====
			\begin{minipage}[t]{0.49\linewidth}
				\centering
				\begin{tikzpicture}[scale=0.5]
					\rvBoard
					\rvCoords
					\rvStonesBlack{E4, D5, E5, E6}
					\rvStonesWhite{D4}
					\rvLastMove{E6}
					\rvFlips{E5}
					\rvMovesWhite{F4, D6, F6}
				\end{tikzpicture}
				\par\vspace{0.2em}{\footnotesize (c) Schwarz zuvor: C6}
			\end{minipage}%
			\hfill
			\begin{minipage}[t]{0.49\linewidth}
				\centering
				\begin{tikzpicture}[scale=0.5]
					\rvBoard
					\rvCoords
					\rvStonesBlack{E4, D5, E5, F5}
					\rvStonesWhite{D4}
					\rvLastMove{F5}
					\rvFlips{E5}
					\rvMovesWhite{F4, D6, F6}
				\end{tikzpicture}
				\par\vspace{0.2em}{\footnotesize (d) Schwarz zuvor: F5}
			\end{minipage}
			
		\end{minipage}%
	}
	\vspace{0.5\baselineskip}
	\caption[Erste regelkonforme Züge bei Othello]{Erste regelkonforme Züge von Schwarz und Weiß im Spiel Othello.}
	\label{fig:othello-initiale-spielsituationen}
 \end{figure}

Die initialen Spielzüge lassen sich gemäß den zuvor eingeführten Definitionen in einen gewurzelten gerichteten Baum überführen, der in Abbildung~\ref{fig:othello-spielbaum-initial} dargestellt ist.
	
\begin{figure}[h!]
	\centering
	
	\forestset{
		Knotenlabel/.style={
			label/.append style={font=\normalsize\bfseries}
		},
		EL/.style n args={1}{
			edge label={
				node[midway, above, sloped,
				fill=white,
				inner sep=1.8pt,
				font=\normalsize\normalfont]{#1}
			}
		}
	}
	
	\begin{forest}
		for tree={
			grow'=south,
			edge={-latex, line width=1pt},
			parent anchor=south,
			child anchor=north,
			s sep=12mm,
			l sep=24mm,
			draw=none,
			circle,
			fill,
			inner sep=2pt,
			label distance=2.5mm,
			Knotenlabel,
		}
		[ , label=above:{(i)}
		[ , label=right:{(d)}, EL={F5}
		[ , EL={C5} ]
		[ , EL={E3} ]
		[ , EL={C3} ]
		]
		[ , label=right:{(c)}, EL={C6}
		[ , EL={C5} ]
		[ , EL={E3} ]
		[ , EL={C3} ]
		]
		[ , label=right:{(b)}, EL={D3}
		[ , EL={F4} ]
		[ , EL={F6} ]
		[ , EL={C6} ]
		]
		[ , label=right:{(a)}, EL={C4}
		[ , EL={F4} ]
		[ , EL={F6} ]
		[ , EL={C6} ]
		]
		]
	\end{forest}
	
	\vspace{0.5\baselineskip}
	\caption[Spielbaum der initialen Züge in Othello]{Spielbaum der initialen Züge in Othello.}
	\label{fig:othello-spielbaum-initial}
\end{figure}

Der dargestellte Spielbaum ist nicht vollständig und enthält daher keine Endknoten, sondern ausschließlich Entscheidungsknoten. Auf der ersten Ebene (der Wurzel) ist die Ausgangssituation dargestellt. Die davon ausgehenden Kanten repräsentieren die möglichen Aktionen von Schwarz. Auf der nächsten Ebene sind die Spielsituationen dargestellt, in denen Weiß am Zug ist. Die Kanten geben die möglichen Folgeaktionen von Weiß an.

Um das Spiel formal als endliches extensives Spiel mit perfekter Information zu modellieren, müssen der vollständige Spielbaum sowie die Spielermenge $I = \{1 = \text{Schwarz},\, 2 = \text{Weiß}\}$ festgelegt werden. Dabei gilt das Aussetzen als zulässige Aktion, sofern kein regelkonformer Zug möglich ist. Der Spielbaum endet, sobald das Spiel abgeschlossen ist, also wenn kein weiterer Zug ausgeführt werden kann. Jedem Endknoten, der eine Endspielsituation beschreibt, wird ein Ergebnis zugeordnet. Diese Ergebnisse werden für jeden Spieler mithilfe einer Nutzenfunktion bewertet, um eine Präferenzrelation zu definieren. So kann einem Sieg der Wert $1$, einer Niederlage der Wert $-1$ und einem Remis der Wert $0$ zugewiesen werden. Alternativ kann die Nutzenfunktion auf der Differenz der Spielsteine basieren.

\subsection{Entscheidungsprinzipien in kompetitiven Spielen}
\label{subsec:prinzipien-kompetitive-spiele}

Wie im vorangegangenen Kapitel~\ref{subsec:modellierung-spiele-baum} dargestellt, handelt es sich bei Othello um ein Spiel, das mithilfe der Spieltheorie formal modelliert werden kann. Das Spiel weist mehrere charakteristische Eigenschaften auf, die für die algorithmische Analyse von zentraler Bedeutung sind:
Othello ist zunächst ein kompetitives Spiel. Die Spieler kommunizieren nicht miteinander, schließen keine bindenden Vereinbarungen und verfolgen strikt antagonistische Ziele. Darüber hinaus ist das Spiel deterministisch, da es keine zufälligen Komponenten enthält und jeder Spielzustand eindeutig aus den zuvor ausgeführten Zügen hervorgeht. Es handelt sich um ein Zwei-Personen-Spiel, in dem die Spieler abwechselnd Züge ausführen.
Weiterhin ist Othello ein Nullsummenspiel, da die Nutzenfunktionen der beiden Spieler entgegengesetzt sind, sodass ein Gewinn des einen Spielers zwangsläufig einen Verlust des anderen impliziert. Schließlich besitzt das Spiel perfekte Information, da beide Spieler jederzeit den vollständigen aktuellen Spielzustand, die Spielregeln sowie alle möglichen Züge des Gegners kennen.

Spiele mit diesen Eigenschaften werden häufig in extensiver Form beschrieben, typischerweise durch die Darstellung als Spielbaum. Dabei zeigt sich, dass die vollständige Lösung eines solchen Spiels rechnerisch anspruchsvoll ist, da der zugehörige Spielbaum eine große Anzahl an Knoten und Kanten umfasst. In der Praxis ist es daher erforderlich, Entscheidungen zu treffen, ohne die optimale Lösung exakt zu kennen oder den gesamten Spielbaum vollständig zu durchsuchen, insbesondere bei begrenzten zeitlichen und rechnerischen Ressourcen. Problemklassen mit diesen Eigenschaften werden in der Algorithmik, insbesondere im Bereich der Künstlichen Intelligenz, als adversariale Suche bezeichnet \autocite[Kap.~5.1]{russellArtificialIntelligenceModern2016}. Im Folgenden werden die grundlegenden Konzepte adversarialer Suchverfahren eingeführt. Die konkreten Algorithmen und deren Anwendung auf das Spiel Othello werden in Kapitel~\ref{cha:stand-der-technik} erläutert.


\todo[inline]{Grundlegende Konzepte/Begriffe adversariale Suche}


% MAX-Knoten:
% - Repräsentieren eigene Entscheidungszustände
% - Auswahl der Aktion mit maximalem Bewertungswert

% MIN-Knoten:
% - Repräsentieren gegnerische Entscheidungszustände
% - Auswahl der Aktion mit minimalem Bewertungswert

% Bewertung:
% - Exakt:
%   * bei vollständiger Durchsuchung bis zu Terminalzuständen
% - Approximiert:
%   * bei begrenzter Suchtiefe mittels Bewertungsfunktion

% Begriffe mit direkter algorithmischer Bedeutung:
% - Verzweigungsfaktor:
%   * Anzahl möglicher Züge pro Zustand
%   * bestimmt exponentielles Wachstum des Suchraums
% - Suchtiefe:
%   * Anzahl der simulierten Halbzüge
% - Cutoff-Tiefe:
%   * maximale Tiefe aufgrund begrenzter Rechenzeit
% - Bewertungsfunktion:
%   * heuristische Abschätzung der Güte nicht-terminaler Zustände
% - Worst-Case-Optimierung:
%   * zentrale Annahme aller Minimax-Varianten

% Motivation für Algorithmuserweiterungen:
% - Vollständige Minimax-Suche:
%   * theoretisch korrekt
%   * praktisch kaum durchführbar
% - Notwendigkeit von:
%   * Suchbaumreduktion
%   * effizienterer Traversierung
%   * heuristischer Entscheidungsunterstützung


\subsection{Spieltheoretische Einordnung von Othello}
\label{subsec:spieltheoretische-einordnung-othello}

% Klassifikation:
% - Zwei-Personen-Spiel
% - Nullsummenspiel
% - Nicht-kooperativ
% - Deterministisch
% - Vollständige Information
% - Endlicher Zustands- und Aktionsraum

% Bedeutung dieser Einordnung:
% - Keine Unsicherheit -> keine probabilistischen Modelle notwendig
% - Keine verdeckten Informationen -> vollständige Spielbaummodellierung möglich
% - Gegensätzliche Ziele -> Minimax-basierte Verfahren geeignet

% Zustandsraumcharakteristik:
% - Sehr große Anzahl möglicher Brettkonfigurationen
% - Hoher durchschnittlicher Verzweigungsfaktor
% - Tiefe Spielbäume bis zum Spielende

% Algorithmische Konsequenzen:
% - Klassische Suchalgorithmen:
%   * prinzipiell korrekt
%   * aber rechnerisch teuer
% - Erweiterungen notwendig:
%   * Alpha-Beta-Pruning zur Reduktion des Suchraums
%   * Iterative Deepening zur besseren Zeitkontrolle
%   * Move Ordering zur Maximierung von Pruning-Effekten
% - Bewertungsheuristiken zwingend:
%   * exakte Bewertung nur am Spielende möglich
%   * heuristische Näherung notwendig für frühe Entscheidungen

% Abgrenzung zu nicht berücksichtigten Verfahren:
% - Verfahren mit:
%   * hohem Speicherbedarf
%   * vielen Simulationen
%   * langer Lernphase
% - Unter Hardwarebeschränkungen ungeeignet

\section{Aufbau und Komponenten von LEGO\textsuperscript{\textregistered} SPIKE}
\label{sec:lego-spike}

\section{Grundlagen robotischer Systeme}
\label{sec:roboter}



