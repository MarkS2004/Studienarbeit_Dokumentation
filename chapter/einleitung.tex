\chapter{Einleitung}
\label{cha:Einleitung}
Spiele dienen seit den Anfängen der Informatik als etablierte Testumgebung für die Entwicklung von Algorithmen, Entscheidungsstrategien und Verfahren der künstlichen Intelligenz. Durch klar definierte Regeln, deterministische Zustandsübergänge und eindeutig messbare Erfolgskriterien ermöglichen sie die systematische Analyse komplexer Entscheidungsprobleme. Insbesondere Brettspiele haben sich dabei als geeignete Abstraktion realer Planungs- und Optimierungsprobleme erwiesen.
Das Strategiespiel Othello nimmt hierbei eine besondere Stellung ein, da es trotz eines einfachen Regelwerks eine hohe kombinatorische Komplexität aufweist und somit ein geeignetes Untersuchungsobjekt der Spieltheorie und algorithmischen Entscheidungsfindung darstellt. Die Übertragung dieses ursprünglich softwarebasierten Problems auf einen physischen Roboter erweitert die Problemstellung um zusätzliche Herausforderungen. Neben spieltheoretischen Aspekten sind insbesondere Wahrnehmung, Zustandsinterpretation sowie Aktionsplanung und -ausführung unter realen Bedingungen zu berücksichtigen. Insbesondere bei der Realisierung mit einem aus LEGO\textsuperscript{\textregistered} Education SPIKE\texttrademark{} aufgebauten System treten Einschränkungen hinsichtlich Rechenleistung, Sensorik und Aktorik in den Vordergrund, die eine angepasste Herangehensweise erfordern.

Ziel dieser Studienarbeit ist die systematische Ausarbeitung eines Gesamtkonzepts für einen solchen Othello spielenden Roboter auf Basis von  LEGO\textsuperscript{\textregistered} Education SPIKE\texttrademark{}. Im Mittelpunkt steht der Entwurf eines konsistenten Systemkonzepts, das geeignete Ansätze zur Spielzustandserfassung, algorithmischen Entscheidungsfindung sowie zur Aktionsplanung und -ausführung kombiniert und dabei die gegebenen Einschränkungen hinsichtlich Rechenleistung, Sensorik und Aktorik berücksichtigt.
Die Arbeit beschränkt sich auf den konzeptionellen Entwurf; eine praktische Umsetzung oder experimentelle Validierung ist nicht Gegenstand dieser Studienarbeit. Das Systemkonzept ist so auszulegen, dass es eine regelkonforme, robuste und grundsätzlich wettbewerbsfähige Spielweise ermöglicht.

Zur Verfügung stand der LEGO\textsuperscript{\textregistered} Education SPIKE\texttrademark{} Prime Baukasten, der ausschließlich verwendet werden darf. Das in dieser Arbeit entwickelte Konzept soll im Rahmen einer folgenden Studienarbeit umgesetzt werden, um gegen weitere Roboter anzutreten und möglichst erfolgreich  zu sein. Hierfür werden zusätzliche Regeln definiert, die die Rahmenbedingungen einschränken. Eine detaillierte Beschreibung der Ausgangssituation erfolgt im weiteren Verlauf der Arbeit.

Zur systematischen Entwicklung eines Konzepts wird der Systems-Engineering-Ansatz angewendet. Zunächst werden die Ausgangssituation analysiert und die Anforderungen an das technische System definiert. Anschließend erfolgt eine funktionale Analyse des Systems. Für die identifizierten Teilfunktionen werden geeignete Lösungsansätze ermittelt und durch eine Nutzwertanalyse strukturiert bewertet. Auf dieser Grundlage entsteht das finale Konzept. Dieses Vorgehen wird durch eine umfangreiche Literaturrecherche sowie die Simulation verschiedener Spielalgorithmen gestützt.

Die vorliegende Arbeit ist wie folgt aufgebaut: In Kapitel \ref{cha:Grundlagen} werden die theoretischen Grundlagen behandelt. Dazu gehören der Spielaufbau und die Regeln von Othello, die Spieltheorie sowie das vorliegende Klemmbaustein-Baukastensystem und die Grundlagen roboterischer Systeme. Darauf aufbauend wird in Kapitel \ref{cha:stand-der-technik} der aktuelle Stand der Algorithmik für Othello dargestellt. In Kapitel \ref{cha:konzeptentwurf} erfolgt der zuvor beschriebene systematische Konzeptentwurf. Das entstandene Konzept wird anschließend in Kapitel \ref{cha:diskussion} kritisch hinterleuchtet, bewertet und diskutiert. Abschließend bietet Kapitel \ref{cha:zusammenfassung} eine Zusammenfassung der Studienarbeit sowie einen Ausblick auf die Umsetzung des Konzepts in einer Folgestudienarbeit.
