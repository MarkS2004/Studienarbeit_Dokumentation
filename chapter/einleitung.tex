\chapter{Einleitung}
\label{cha:Einleitung}
Spiele dienen seit den Anfängen der Informatik als etablierte Testumgebung für die Entwicklung von Algorithmen, Entscheidungsstrategien und Verfahren der künstlichen Intelligenz. Durch klar definierte Regeln, deterministische Zustandsübergänge und eindeutig messbare Erfolgskriterien ermöglichen sie die systematische Analyse komplexer Entscheidungsprobleme. Insbesondere Brettspiele haben sich dabei als geeignete Abstraktion realer Planungs- und Optimierungsprobleme erwiesen.
Das strategische Brettspiel Othello nimmt hierbei eine besondere Stellung ein, da es trotz eines einfachen Regelwerks eine hohe kombinatorische Komplexität aufweist und somit ein geeignetes Untersuchungsobjekt der Spieltheorie und algorithmischen Entscheidungsfindung darstellt. Die Übertragung dieses ursprünglich softwarebasierten Problems auf einen physischen Roboter erweitert die Problemstellung um zusätzliche Herausforderungen. Neben spieltheoretischen Aspekten sind insbesondere Wahrnehmung, Zustandsinterpretation sowie Aktionsplanung und -ausführung unter realen Bedingungen zu berücksichtigen. Insbesondere bei der Realisierung mit einem aus LEGO\textsuperscript{\textregistered} SPIKE aufgebauten System treten Einschränkungen hinsichtlich Rechenleistung, Sensorik und Aktorik in den Vordergrund, die eine angepasste Herangehensweise erfordern.
\\
Ziel dieser Studienarbeit ist die systematische Ausarbeitung eines Gesamtkonzepts für einen solchen Othello spielenden Roboter auf Basis von LEGO\textsuperscript{\textregistered} SPIKE. Im Mittelpunkt steht der Entwurf eines konsistenten Systemkonzepts, das geeignete Ansätze zur Spielzustandserfassung, algorithmischen Entscheidungsfindung sowie zur Aktionsplanung und -ausführung kombiniert und dabei die gegebenen Einschränkungen hinsichtlich Rechenleistung, Sensorik und Aktorik berücksichtigt.
Die Arbeit beschränkt sich auf den konzeptionellen Entwurf; eine praktische Umsetzung oder experimentelle Validierung ist nicht Gegenstand dieser Studienarbeit. Das Systemkonzept ist so auszulegen, dass es eine regelkonforme, robuste und grundsätzlich wettbewerbsfähige Spielweise ermöglicht.
\todo[inline]{Leitfragen, Ausgangslage(was steht zur Verfügung, es mussten gemeinsam Regeln definiert werden), geplante Vorgehensweise, Methoden zur Bearbeitung, Kurzübersicht über die Inhalte der Kapitel}





