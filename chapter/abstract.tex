\chapter*{Kurzfassung}
\phantomsection\pdfbookmark{Kurzfassung}{kurzfassung}

Die vorliegende Arbeit beschäftigt sich mit der Erstellung eines Konzepts für ein Robotersystem zur Ausführung des Spiels Othello auf einer digitalen Plattform. Die Problemstellung ergibt sich aus der Herausforderung, ein System zu entwickeln, das in der Lage ist, das Spiel selbstständig regelkonform zu spielen, dabei präzise Bewegungen auf einem Tablet auszuführen und die begrenzten Ressourcen der verwendeten LEGO\textsuperscript{\textregistered} Education SPIKE\texttrademark{} Prime Hardware effizient zu nutzen. Zusätzlich muss ein Sicherheitsbereich eingehalten und die Interaktion mit der digitalen Schachuhr zur Spielzugübergabe zuverlässig realisiert werden.

Ziel der Arbeit war es, ein Gesamtkonzept zu entwickeln, das sowohl die Grundlage der mechanische Gestaltung des Roboters als auch die algorithmische Planung der Spielzüge umfasst. Das Konzept soll als Grundlage für die spätere praktische Umsetzung dienen und die Einhaltung der festgelegten Anforderungen sicherstellen.  

Für die Erarbeitung des Konzepts wurde ein Systems-Engineering-Ansatz gewählt. Zunächst werden die Rahmenbedingungen und Ausgangssituation analysiert, anschließend die funktionalen und nicht-funktionalen Anforderungen sowie Randbedingungen definiert. Darauf aufbauend erfolgte eine funktionale Zerlegung des Systems in klar abgegrenzte Teilfunktionen, darunter die Erkennung des Startbefehls, die Erfassung des Spielzustands, die Bestimmung des nächsten Spielzugs, die präzise Ausführung des Spielzugs, die Rückfahrt in die Ruheposition und die Beendigung des Spielzugs. Für jede Teilfunktion wurden verschiedene Lösungsansätze entwickelt, bewertet und diskutiert, die zusammen ein Gesamtkonzept ergeben.  

Die Arbeit liefert ein umfassendes theoretisches Konzept, das die Auswahl eines SCARA-ähnlichen Roboters für die präzise Bewegungssteuerung, die Verwendung des Negamax Algorithmus für die Spielzugplanung und eine strukturierte Handhabung der Spielumgebung umfasst. Das Konzept stellt sicher, dass der Roboter die definierten Anforderungen erfüllen kann und bildet die Grundlage für die praktische Umsetzung.


\clearpage

\chapter*{Abstract} %*-Variante sorgt dafür, das Abstract nicht im Inhaltsverzeichnis auftaucht
\phantomsection\pdfbookmark{Abstract}{abstract}

This thesis deals with the creation of a concept for a robot system to play the game Othello on a digital platform. The problem arises from the challenge of developing a system that is capable of playing the game independently in accordance with the rules, while executing precise movements on a tablet and making efficient use of the limited resources of the LEGO\textsuperscript{\textregistered} Education SPIKE\texttrademark{} Prime hardware used. In addition, a safety area must be maintained and interaction with the digital chess clock for handing over moves must be reliably implemented.

The aim of the thesis was to develop an overall concept that encompasses both the basis of the mechanical design of the robot and the algorithmic planning of the moves. The concept is intended to serve as a basis for later practical implementation and to ensure compliance with the specified requirements. 

A systems engineering approach was chosen for the development of the concept. First, the framework conditions and initial situation were analyzed, then the functional and non-functional requirements and boundary conditions were defined. Based on this, the system was functionally broken down into clearly defined sub-functions, including recognition of the start command, recording of the game status, determination of the next move, precise execution of the move, return to the rest position, and termination of the move. For each sub-function, various solutions were developed, evaluated, and discussed, which together form an overall concept.  

The work provides a comprehensive theoretical concept that includes the selection of a SCARA-like robot for precise motion control, the use of the Negamax algorithm for move planning, and structured handling of the game environment. The concept ensures that the robot can meet the defined requirements and forms the basis for practical implementation.

\clearpage