\chapter{Zusammenfassung und Ausblick}
\label{cha:zusammenfassung}

Ziel der Arbeit war die Entwicklung eines Gesamtkonzepts für ein Robotersystem, das Othello auf einer digitalen Plattform spielen kann. Dazu wurden die theoretischen Grundlagen des Spiels erläutert und adversarielle Suche sowie allgemeine Entscheidungsprinzipien anhand von Beispielen vorgestellt und darauf angewandt, um das Spiel einzuordnen. Grundlage der Hardware bildet das LEGO\textsuperscript{\textregistered} Education SPIKE\texttrademark{} Prime Baukastensystem, dessen Grundkomponenten erklärt und deren Bedeutung für das Konzept hervorgehoben wurden. Für das Zusammenspiel dieser Komponenten sind die Prinzipien robotischer Systeme relevant, weshalb diese ebenfalls aufgeschlüsselt und anhand bedeutsamer Beispiele diskutiert wurden. Für die logische Steuerung des Roboters im Hinblick auf die Spiellogik wurden Suchalgorithmen erläutert und speziell für Othello untersucht, um einen passenden Algorithmus für die spätere Umsetzung auszuwählen. Außerdem wurde das Ausführungskonzept nach dem Systems-Engineering-Ansatz entworfen. Dieser umfasst die festgelegten Rahmenbedingungen der Aufgabenstellung sowie eine detaillierte Anforderungsanalyse, um die zu realisierenden Funktionen zu definieren und zu klassifizieren. Für den konkreten funktionalen Ablauf wurden auf Basis der zuvor beschriebenen Prinzipien Lösungsansätze diskutiert und ein geeigneter ausgewählt, um das Konzept abzurunden. Zusätzlich wurden mögliche Probleme aufgezeigt, die bei der Umsetzung des Konzepts auftreten könnten.

Im zweiten Teil der Studienarbeit liegt der Schwerpunkt auf der praktischen Umsetzung des entwickelten Konzepts. Ziel ist es, den theoretisch entworfenen Roboteraufbau und die implementierten Algorithmen auf dem LEGO\textsuperscript{\textregistered} Education SPIKE\texttrademark{} Prime System zu realisieren und die autonome Spielausführung unter realen Bedingungen zu testen. Dabei wird untersucht, inwieweit die zuvor simulierten oder theoretisch berechneten Abläufe in der Praxis reproduzierbar sind und welche Anpassungen notwendig werden, um die Zuverlässigkeit und Effizienz des Systems zu gewährleisten.

\clearpage

Die praktische Umsetzung beginnt mit dem mechanischen Aufbau des Roboters gemäß dem entworfenen SCARA-ähnlichen Konzept. Besonderes Augenmerk liegt auf der Feinabstimmung der Gelenke, um die geforderte Wiederholgenauigkeit von zu erreichen. Parallel dazu soll die Software auf dem Controller implementiert werden, einschließlich der Integration des Negamax-Algorithmus zur Zugberechnung, der Steuerung der Bewegungsabläufe und Interaktion mit dem Tablet. Im selben Rahmen wird überprüft, ob die Bewegungsausführung, Eingabe der Spielzüge und die Einhaltung der Sicherheitszone korrekt durchgeführt werden. Des weiteren wird die genaue Zugdauer sowie der Speicherverbrauch des gesamten Systems erfasst.

Darüber hinaus bietet der praktische Teil die Möglichkeit, den Einfluss von Umgebungsvariablen zu bewerten, etwa unterschiedliche Lichtverhältnisse auf dem Tablet oder geringfügige Abweichungen in der Spielfeldpositionierung. Diese Beobachtungen ermöglichen es, die Robustheit des Roboters zu steigern und Anpassungen vorzunehmen, die in der theoretischen Planung noch nicht berücksichtigt werden konnten.

%Auf zwei bis drei Seiten soll auf folgende Punkte eingegangen werden:
%
%\begin{itemize}
%	\item Welches Ziel sollte erreicht werden
%	\item Welches Vorgehen wurde gewählt
%	\item Was wurde erreicht, zentrale Ergebnisse nennen, am besten quantitative Angaben machen
%	\item Konnten die Ergebnisse nach kritischer Bewertung zum Erreichen des Ziels oder zur Problemlösung beitragen
%	\item  Ausblick
%\end{itemize}
%
%In der Zusammenfassung sind unbedingt klare Aussagen zum Ergebnis der Arbeit zu nennen. Üblicherweise können Ergebnisse nicht nur qualitativ, sondern auch quantitativ benannt werden, z.~B. \glqq \ldots konnte eine Effizienzsteigerung von \SI{12}{\percent} erreicht werden.\grqq~oder \glqq \ldots konnte die Prüfdauer um \SI{2}{\hour} verkürzt werden\grqq.
%
%Die Ergebnisse in der Zusammenfassung sollten selbstverständlich einen Bezug zu den in der Einleitung aufgeführten Fragestellungen und Zielen haben.
